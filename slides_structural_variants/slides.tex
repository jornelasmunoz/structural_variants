\documentclass[10pt, t]{beamer}
\usetheme{Madrid}
\documentclass{article}
\usepackage{spconf,amsmath, amssymb}
\usepackage{xcolor, graphicx}
\usepackage{mathtools} % for constraints matrix
% Example definitions.
% --------------------
\def\x{{\mathbf x}}
\def\L{{\cal L}}

% signal vectors
\def\R{\mathbb{R}} %Real Numbers
\newcommand{\fP}{\vec{f}_P }
\newcommand{\fC}{\vec{f}_C }
\newcommand{\fH}{\vec{f}_H }
\newcommand{\fN}{\vec{f}_N }
\newcommand{\sP}{\vec{s}_P }
\newcommand{\sC}{\vec{s}_C }
\newcommand{\sH}{\vec{s}_H }
\newcommand{\sN}{\vec{s}_N }
\newcommand{\zP}{\vec{z}_P }
\newcommand{\zC}{\vec{z}_C }
\newcommand{\zH}{\vec{z}_H }
\newcommand{\zN}{\vec{z}_N }
\newcommand{\yP}{\vec{y}_P }
\newcommand{\yC}{\vec{y}_C }
\newcommand{\yH}{\vec{y}_H }
\newcommand{\yN}{\vec{y}_N }



\title[Structural Variant Detection]{Sparse Negative Binomial Signal Recovery for Genomic Variant Prediction in Diploid Species}
\author[J. Ornelas Munoz]{\textbf{Jocelyn Ornelas Munoz$^{\star}$}, Erica M. Rutter$^{\star}$, Mario Banuelos$^{\dagger}$, Suzanne S. Sindi$^{\star}$,  Roummel F. Marcia$^{\star}$}

\institute[]{\vspace{-4mm} \\ \normalsize{$^*$Department of Applied Mathematics, UC Merced\\
$^\dagger$ Department of Mathematics California State University, Fresno}\\ 	
	%\vspace{-2mm} 
% 	\begin{table}[h!]
% 		\begin{center}
% 			%\resizebox{0.6\textwidth}{!}{
% 			\resizebox{0.4\textwidth}{!}{
% 				\begin{tabular}{rl}
% 					Advisor:        	&   Dr.\ Roummel F. Marcia \\
% 					                    &   Dr.\ Erica Rutter
% 				% 	Committee Members:  &    Dr.\ Omar DeGuchy (LLNL)\\
% 				% 	&     Dr.\ Arnold Kim (Applied Math)\\ 
% 				% 	&     Dr.\ Chrysoula Tsogka (Applied Math) \\
% 			\end{tabular}}
% 		\end{center}
% 	\end{table}
	\vspace{2mm}
%	\small{Group Meeting} 
	\vspace{-3mm}
}

\date{Wednesday, November 16, 2022}

\titlegraphic{\vspace{-5pt}
	\includegraphics[width=1.5cm]{images/UCM_Seal.png}
}

\begin{document}

\begin{frame}
  \titlepage
\end{frame}

%\AtBeginSection[]{
%  \begin{frame}\frametitle{Outline}
%  \tableofcontents[currentsection]
%  \end{frame}
%}

% \begin{frame}\frametitle{Overview}
%  \tableofcontents[]
% \end{frame}
%
%\AtBeginSection[] {
%\begin{frame}{Outline}
%		\tableofcontents[currentsection,currentsubsection]
%\end{frame}}
%
%\AtBeginSubsection[] {
%	\begin{frame}{Outline}
%		\tableofcontents[currentsection,currentsubsection]
%\end{frame}}

% \begin{frame}{Overview}
% \bigskip
% \textbf{I.} \? \? \? \textbf{Previous Work}\\
% \bigskip
% \textbf{II.} \? \? \textbf{Ongoing Work}\\
% \bigskip
% \textbf{III.}  \? \textbf{Future Work}\\
% \bigskip
% \textbf{IV.} \? \textbf{Accomplishments}\\
% \bigskip
% \textbf{V.} \? \? \textbf{Timeline} 
% \end{frame}
%-------------------------------------------------------------------
%                    INTRODUCTION
%-------------------------------------------------------------------
\section{Overview}
\begin{frame}{Overview}
%Encoded in our DNA are the instructions as to how 

% Genomic disorders are caused by changes in the human genome --e.g. cancer, down syndrome, autism, etc. In particular, \textcolor{blue}{structural variants (SVs)} are 
% \begin{itemize}
%     \item Insertions, deletions, inversions, duplications, and transpositions in genomic sequences
%     \item Large-scale changes in the DNA sequence ($~50-200$ letters)
%     \item Rare, hard to detect occurrences in the genome
% \end{itemize}

Goal: Reconstruct signals from noisy measurements as accurately as possible. 

\end{frame}

%-------------------------------------------------------------------
%                    PROBLEM FORMULATION
%-------------------------------------------------------------------

\section{Problem Formulation}
\begin{frame}{Problem Formulation}
Suppose we have sequencing data for two individuals. \\

\begin{figure}[h]
 	 	\includegraphics[width = .65\textwidth]{imgs/SVs.png}
\end{figure}
We denote the true signals:
\medskip
\begin{itemize}
    \item Parent : $ \fP  \in \{0,1,2\}^n $
    \item Child : $\fC = \fH +\fN \in \{0,1,2\}^{n}$
\end{itemize}
\medskip
where $\fH \in \{0,1,2\}^n$ and $\fN \in \{0,1,2\}^n$ correspond to the vectors of inherited ($H$) and novel ($N$) SVs in the child, respectively.
\end{frame}

\begin{frame}{Problem Formulation}
    For each individual signal we consider two indicator signals such that $\vec{f}_i = 2\vec{z}_i + \vec{y}_i$ for $i \in \{P, H, N\}$. \ %\cite{MB_diploidTrios}.
{\begin{center}
	\begin{columns}
		\begin{column}{0.3\textwidth}
		Homozygous indicator: 
			\vspace{3mm}
			\scalebox{0.8}{
				\begin{tabular}{cl}
					$\vec{z}_j = 1$  & if 2 copies of an SV are present at location $j$ \medskip \\  
					$\vec{z}_j = 0$ & otherwise
				\end{tabular}
			}
		Heterozygous indicator: 
			\vspace{5mm}
			\scalebox{0.8}{
				\begin{tabular}{cl}
					$\vec{y}_j = 1$  & if 1 copy of an SV is present at location $j$ \medskip \\
					$\vec{y}_j = 0$ & otherwise
				\end{tabular}
			}
		\end{column}
		\hspace{20mm}
		\begin{column}{0.5\textwidth}
			\vspace{-5mm}
			\begin{figure}
				\includegraphics[width=.8\linewidth]{imgs/indicator_vecs.png}
			\end{figure}
		\end{column}
	\end{columns}
	\end{center}}
\end{frame}

\begin{frame}{Problem Formulation}
	Next, we split the child's diploid indicators into inherited, $H$, and novel, $N$, indicators 
	{\begin{center}
			\begin{columns}
				\begin{column}{0.49\textwidth}
					\begin{figure}
						\includegraphics[width=.9\linewidth]{imgs/parent_z_y.png}
					\end{figure}
				\end{column}
				%\hspace{5mm}
				\begin{column}{0.49\textwidth}
					\vspace{-5mm}
					\begin{figure}
						\includegraphics[width=.9\linewidth]{imgs/child_z_y_all.png}
					\end{figure}
				\end{column}
			\end{columns}
	\end{center}}
\end{frame}

\begin{frame}{Problem Formulation}
Next, we split the child's diploid indicators into inherited, $H$, and novel, $N$, indicators 
	{\begin{center}
			\begin{columns}
				\begin{column}{0.49\textwidth}
					\begin{figure}
						\includegraphics[width=.9\linewidth]{imgs/parent_z_y_arrows.png}
					\end{figure}
				\end{column}
				%\hspace{5mm}
				\begin{column}{0.49\textwidth}
					\vspace{-5mm}
					\begin{figure}
						\includegraphics[width=.9\linewidth]{imgs/child_z_y_arrows.png}
					\end{figure}
				\end{column}
			\end{columns}
	\end{center}}
\end{frame}

\begin{frame}{Problem Formulation}
	We leverage relatedness in our model by incorporating the following assumptions:
	\pause
	\begin{itemize}
		\item An individual can only have 0, 1, or 2 copies of an SV per candidate location
		\pause
		
		\item 	If the child has an inherited SV,  the SV cannot also be novel
		\pause
		
		\item 	If the child has an inherited homogenous SV, the parent has at least a heterogeneous SV 
		\pause
		
		\item 	If the parent has a homogeneous SV, the child must have at least a heterogeneous SV
		\pause
		
		\item 	If the child has a novel SV, the parent does not have that SV
		
	\end{itemize}
\pause
Mathematically, these are written as constraints:
%\renewcommand{\arraystretch}{1.3}
\begin{align*}
	\mathcal{S} = \left\{
	\vec{f} = 
	\begin{bmatrix}
		\zP \\ \zH \\ \zN \\ \yP \\ \yH \\ \yN 
	\end{bmatrix} \in \R^{6n} : 
	\begin{matrix*}[l]
		&\mathbf{0} \leq \, \vec{z}_i + \vec{y}_i \, \leq \mathbf{1}\\
		& \mathbf{0} \leq \, \zH + \yH + \zN + \yN \, \leq \mathbf{1}\\
		&\mathbf{0} \leq \, \zH   \, \leq \, \zP + \yP  \, \leq \mathbf{1}\\
		&\mathbf{0} \leq \, \zP   \, \leq \, \zH + \yH  \, \leq \mathbf{1}\\
		&\mathbf{0} \leq \, \zN + \yN   \, \leq \, 1- (\zP +\yP)  \, \leq \mathbf{1}
	\end{matrix*}
	\right\}
\end{align*}

	
\end{frame}
%\begin{frame}{Problem Formulation}
%    We will denote the set of all vectors satisfying the constraints by $\mathcal{S}$,
%    \renewcommand{\arraystretch}{1.3}
%    \begin{align*}
%        \mathcal{S} = \left\{
%            \vec{f} = 
%            \begin{bmatrix}
%                \zP \\ \zH \\ \zN \\ \yP \\ \yH \\ \yN 
%            \end{bmatrix} \in \R^{6n} : 
%            \begin{matrix*}[l]
%            &\mathbf{0} \leq \, \vec{z}_i + \vec{y}_i \, \leq \mathbf{1}\\
%            & \mathbf{0} \leq \, \zH + \yH + \zN + \yN \, \leq \mathbf{1}\\
%            &\mathbf{0} \leq \, \zH   \, \leq \, \zP + \yP  \, \leq \mathbf{1}\\
%            &\mathbf{0} \leq \, \zP   \, \leq \, \zH + \yH  \, \leq \mathbf{1}\\
%            &\mathbf{0} \leq \, \zN + \yN   \, \leq \, 1- (\zP +\yP)  \, \leq \mathbf{1}
%            \end{matrix*}
%             \right\}
%    \end{align*}
%\end{frame}
\begin{frame}{Observational Model}
	Under the Negative Binomial assumption, we can model the noise in our observation as \vspace{2em}
	\begin{equation*}
		\begin{bmatrix}
			\vec{s}_P \\
			\vec{s}_C
		\end{bmatrix}
		\sim \text{NegBin}
		\Bigg( \begin{bmatrix}
			\vec{z}_P \,(2 \lambda_P - \varepsilon)\; +\; \vec{y}_P \,(\lambda_P - \varepsilon) \\
			\vec{z}_H \,(2 \lambda_C - \varepsilon)\; +\; \vec{y}_H \,(\lambda_C - \varepsilon) \;+ \; \vec{z}_N \,(2 \lambda_C - \varepsilon) \;+ \;\vec{y}_N \,(\lambda_C - \varepsilon) \\
		\end{bmatrix} \Bigg)
	\end{equation*}
\vspace{2em}

where \smallskip
\begin{itemize}
	\item $\vec{s}_P, \; \vec{s}_C \in \mathbb{R}^n$: length-$n$ vector of counts for the parent and the child, respectively
	\item $\lambda_P, \, \lambda_C$ : sequencing coverage of the parent and the child, respectively
	\item $\varepsilon >0$: measurement errors incurred in the sequencing and mapping process
	
\end{itemize}
\end{frame}


\begin{frame}{Observational Model}
More generally, if we let  
$$\vec{s} = \begin{bmatrix}
	\vec{s}_P \\ \vec{s}_C
\end{bmatrix}, \quad 
\vec{z} = \begin{bmatrix}
	\vec{z}_P \\ \vec{z}_H \\ \vec{z}_N
\end{bmatrix}, \quad 
\vec{y} = \begin{bmatrix}
	\vec{y}_P \\ \vec{y}_H \\ \vec{y}_N
\end{bmatrix}, \quad  
\vec{f} = \begin{bmatrix}
	\vec{z} \\ \vec{y}
\end{bmatrix}$$

we can express our observation model as
$$\vec{s} \sim \text{NegBin}(A \vec{f} + \varepsilon \mathbf{1)}$$
where
\begin{itemize}
	\item $\vec{s} \in \mathbb{Z}_{+}^{2n}$ is a length-$2n$ vector of observed coverage counts
	\item $\vec{f} \in \{0,1\}^{6n}$ is the signal of interest
	\item $\mathbf{1} \in \mathbb{R}^{2n}$ is the vector of ones
	\item $A \in \mathbb{R}^{2n \times 6n}$ is the sequence coverage matrix
\end{itemize}

\end{frame}

\begin{frame}{Observational Model}
	
	The probability of observing a particular vector of counts $\vec{s}$ given the true signal $\vec{f}$ is given by
	\medskip 
	$$p(\vec{s} \; | A\vec{f})\; = \prod_{l=1}^{2n} \left( \frac{1}{1+ (A \vec{f})_l + \varepsilon} \right) \left(  \frac{((A \vec{f})_l + \varepsilon)}{1+ (A \vec{f})_l + \varepsilon} \right)^{s_l} $$
	
	\medskip
	We use gradient-based maximum likelihood approach to recover the indicator variables $\vec{z}_i, \, \vec{y}_i, \; i \in \{\ P,H,N\}$. 
\end{frame}

\begin{frame}{Observational Model}
	
	The probability of observing a particular vector of counts $\vec{s}$ given the true signal $\vec{f}$ is given by
	\medskip 
	$$p(\vec{s} \; | A\vec{f})\; = \prod_{l=1}^{2n} \left( \frac{1}{1+ (A \vec{f})_l + \varepsilon} \right) \left(  \frac{((A \vec{f})_l + \varepsilon)}{1+ (A \vec{f})_l + \varepsilon} \right)^{s_l} $$
	
	\medskip
	We want to minimize the Negative Binomial negative log-likelihood function
	$$F(\vec{f}) \equiv \sum_{l=1}^{2n}  (1 + s_l)\log \big(1+ e_l^T A \vec{f} + \varepsilon \big) - s_l \log \big( e_l^T A \vec{f} + \varepsilon \big)$$
\end{frame}

\begin{frame}{Optimization Framework}
Our objective function takes the following form:

\begin{equation*} \label{minimizationProblem}
    \begin{aligned}
        & \underset{\vec{f} \in \R^{6n}}{\text{minimize}}
        & & F(\vec{f}) + \tau \text{pen}(\vec{f}) \\
        & \text{subject to}
        & & \vec{f} \in \mathcal{S}
    \end{aligned}
\end{equation*}

where 
\begin{itemize}
    \item $F(\vec{f})$ is the Negative Binomial negative log-likelihood function
    \item $\text{pen}(\vec{f}) = (\|\zP\|_1 + \|\zH\|_1 + \|\yP\|_1 + \|\yH\|_1) + \gamma (\|\zN\|_1 + \|\yN\|_1)$ is the $\ell_1$ penalty term
    \item $\tau > 0 $ and $\gamma \gg 1$ are regularization parameters
    \item $\mathcal{S}$ is the set of vectors satisfying the biological constraints
\end{itemize}
\end{frame}

\begin{frame}{Optimization Framework}
 We use a second-order Taylor series expansion to approximate $F(\vec{f})$ at iterate $\vec{f}^k$ and solve the separable quadratic subproblem,
 \smallskip
 \begin{equation} \label{subproblemIterate}
 	\vec{f}^{k+1} = 
 	\begin{aligned}
 		& \underset{\vec{f} \in \R^{6n}}{\text{arg min}}
 		& & \mathcal{Q}(\vec{f})= \frac{1}{2}\|\vec{f} - \vec{r}^{\,k}\|_2^2 + \frac{\tau}{\alpha_k} \text{pen}(\vec{f}) \\
 		& \text{subject to}
 		& & \vec{f} \in \mathcal{S}
 	\end{aligned}
 \end{equation}

\smallskip
where $\vec{r}^{\,k} = 
[\vec{r}_{z_P}^{\,k} ,\vec{r}_{z_H}^{\,k},\vec{r}_{z_N}^{\,k},\vec{r}_{y_P}^{\,k},\vec{r}_{y_H}^{\,k},\vec{r}_{y_N}^{\,k}]^T
= \vec{f}^k $ $- \frac{1}{\alpha_k} \nabla F(\vec{f}^k)$ and $\alpha_k$ is \textcolor{red}{INSERT HERE}.

% ADD FOOTNOTE: (Banuelos et. al., 2018, Harmany, et.al., 2012)
\end{frame}

\begin{frame}{Optimization Framework}
	Our objective function $\mathcal{Q}(\vec{f})$ is separable and decouples into the function
	%\begin{equation*}
	$\mathcal{Q}(\vec{f}) = \sum_{j=1}^{n} \mathcal{Q}_j(\zP, \zH, \zN, \yP, \yH, \yN),$
	%\end{equation*}
	where
			\begin{align*}
			\mathcal{Q}_j(\zP, \zH, \zN, \yP, \yH, \yN) = 
				&\frac{1}{2}\Bigg\{
				((\zP - \vec{r}_{\zP}^{\, k})_j)^2 + 
				((\zH - \vec{r}_{\zH}^{\, k})_j)^2 +
				((\zN - \vec{r}_{\zN}^{\, k})_j)^2 \\ 
				&+ 
				((\yP - \vec{r}_{\yP}^{\, k})_j)^2 + 
				((\yH - \vec{r}_{\yH}^{\, k})_j)^2 +
				((\yN - \vec{r}_{\yN}^{\, k})_j)^2 \Bigg\} \\
				&+
				\frac{\tau}{\alpha_k} \Big\{ |(\zP)_j| + 
				|(\zH)_j| + \gamma |(\zN)_j| \\
				&+ 
				|(\yP)_j| +  |(\yH)_j| + \gamma |(\yN)_j|\Big\}
			\end{align*}
\end{frame}

\begin{frame}{Optimization Framework}
	Since the bounds that define the region $\mathcal{S}$ are component-wise, then  \Cref{subproblemIterate} separates into subproblems of the form: 
			\begin{equation} \label{scalarSubprob}
				\begin{aligned}
					\vec{f}^{k+1}= %\arg
					\min_{\substack{z_P,z_H,z_N,\\ y_P,y_H,y_N \in \R}}
				    &\frac{\tau}{\alpha_k}\; \Big\{ 
					|(\zP)_j| + |(\zH)_j| + \gamma |(\zN)_j| %\\
					%\hspace{10pt}  & +
					+	|(\yP)_j| + |(\yH)_j| + \gamma |(\yN)_j|\Big\} \\
					& + \frac{1}{2} \Big\{
					((\zP - \vec{r}_{\zP}^{\, k})_j)^2 + 
					((\zH - \vec{r}_{\zH}^{\, k})_j)^2 +
					((\zN - \vec{r}_{\zN}^{\, k})_j)^2 \\ & + 
					((\yP - \vec{r}_{\yP}^{\, k})_j)^2 + 
					((\yH - \vec{r}_{\yH}^{\, k})_j)^2 +
					((\yN - \vec{r}_{\yN}^{\, k})_j)^2 \Big\} \\
					& \text{subject to } \vec{f} \in S
				\end{aligned}
			\end{equation}
		where $f_i$ and $r_{z_i}, r_{y_i}$ %for $i \in \{P,H,N\},\,$ 
		are scalar components of $\vec{f_i}$ and $\vec{r}_{z_i}, \vec{r}_{y_i}$, respectively, at the same location.  
\end{frame}
\begin{frame}{Optimization Approach}
	We solve our problem using an alternating block-coordinate descent approach, following the methods. We fix all but one individual and solve \Cref{subproblemIterate} over both indicator variables for that individual.\\
	\medskip
	\begin{tabular}{c c}
		\textbf{Step 0:} & \text{Compute the unconstrained minimizer of \Cref{subproblemIterate}  given by}\\
								   & $\vec{f} = [\vec{r}_{z_P} - \frac{\tau}{\alpha_k} \mathbf{1}_n, \; \vec{r}_{z_H} - \frac{\tau}{\alpha_k} \mathbf{1}_n, \; \vec{r}_{z_N} - \frac{\tau}{\alpha_k}  \gamma \mathbf{1}_n, \; $\\
								   & \hspace{2.5em}$\vec{r}_{y_P} - \frac{\tau}{\alpha_k} \mathbf{1}_n, \; \vec{r}_{y_H} - \frac{\tau}{\alpha_k} \mathbf{1}_n, \; \vec{r}_{y_N} - \frac{\tau}{\alpha_k} \gamma \mathbf{1}_n]^T$\\
								   & \hspace{-12em}where $\mathbf{1}_n \in \R^n$\\
								   & \phantom{ }\\
								   & \hspace{-15em} Initialize indicator variables\\
								   & \phantom{ }\\
								   & %\begin{equation*}
								   	$\begin{aligned}
								   		\hat{z}_P^{(0)} &= r_{z_P}^k - \frac{\tau}{\alpha_k} 						  	 & \quad \hat{y}_P^{(0)} &= r_{y_P}^k - \frac{\tau}{\alpha_k}, \\
								   		\hat{z}_H^{(0)} & = \{0,r_{z_H}^k - \frac{\tau}{\alpha_k},1 \} 			     & \quad \hat{y}_H^{(0)} &= \{0,r_{y_H}^k - \frac{\tau}{\alpha_k},1 \}, \\
								   	    \hat{z}_N^{(0)} &= \{0,r_{z_N}^k - \frac{\tau}{\alpha_k}\gamma,1 \}  & \quad \hat{y}_N^{(0)} &= \{0,r_{y_N}^k - \frac{\tau}{\alpha_k}\gamma,1 \}
								   	\end{aligned}$\\
								   %\end{equation*}\\
	\end{tabular}
	
\end{frame}

\begin{frame}{Optimization Approach}
	(cont.)\\
	\medskip
	\begin{tabular}{c l}
		\textbf{Step 1:}  & Project $\hat{z}_P^{(k-1)}$ and $\hat{y}_P^{(k-1)}$ onto the feasible set $S$ with fixed \\
								   &  inherited and novel variables to obtain  $\hat{z}_P^{(k)}$ and $\hat{y}_P^{(k)}$.\\
								   & \phantom{ }\\ \pause
		\textbf{Step 2:} & Project $\hat{z}_H^{(k-1)}, \hat{y}_H^{(k-1)}$ onto the feasible set $S$ with fixed \\
								   &  parent and child's novel indicator variables to obtain  $\hat{z}_H^{(k)}$ and $\hat{y}_H^{(k)}$.\\
								   & \phantom{ }\\ \pause
		\textbf{Step 3:} & Project $\hat{z}_N^{(k-1)}, \hat{y}_N^{(k-1)}$ onto the feasible set $S$ with fixed \\
								   & parent and child's inherited indicator variables to obtain  $\hat{z}_N^{(k)}$ and $\hat{y}_N^{(k)}$.\\
								   & \phantom{ }\\ \pause
		\textbf{Step 4:} & Repeat Steps $1, 2,$ and $3$ until the relative difference between \\
								   & consecutive iterates converges to $\|\vec{f}^{k+1} - \vec{f}^{k}\|/\|\vec{f}^{k}\| \leq 10^{-8}$.				
	\end{tabular}
	
\end{frame}

\begin{frame}{Optimization Approach}
Insert figure of feasible regions (or do it by steps)
\end{frame}

\begin{frame}{Numerical Experiments}
We implemented our method by  modifying the existing SPIRAL approach to include the negative binomial statistical method.
We compared the Poisson-based predictions (SPIRAL) with the Negative Binomial-based predictions (NEBULA).

%we simulated two parent signals of size $10^5$ with a set number of structural variants and a set similarity of $80\%$ between the parent signals \cite{MB_diploidTrios, MB_SingleParentDiploid}. In the parent signals, $5000$ locations were chosen at random to be structural variants. We then constructed the child signal by first applying a logical implementation of inheritance to $\lfloor 5000p \rfloor$ randomly selected parent structural variants (where $p$ is the percent overlap between parent and child SVs). Next, we chose  $(5000 - \lfloor 5000p \rfloor)$ locations from the remaining $(10^5 - 5000)$ that were not chosen as a parent variant to be novel variants in the child.  After forming the true signals for each individual, the observed signals were generated by sampling from the Negative Binomial distribution with a given coverage and error. For the purpose of testing the proposed approach, only one parent signal was used.
\end{frame}
%\begin{frame}{Numerical Experiments}
%    \begin{itemize}
%        \item Individual signal size: $n = 10^5$
%        \item Total SVs: $k=50$
%        \item Coverage: $\lambda_p = 2$, $\lambda_c = 4$ 
%    \end{itemize}
%    \begin{figure}
% 	 	\includegraphics[width = .75\textwidth]{imgs/table.png}
% 	 	\caption{Table of vector values for resulting reconstructions}
%	 \end{figure}
%\end{frame}
%
%\begin{frame}{Numerical Experiments}
%    \begin{itemize}
%        \item Individual signal size: $n = 10^5$
%        \item Total SVs: $k=50$
%        \item Coverage: $\lambda_p = 2$, $\lambda_c = 4$ 
%    \end{itemize}
%    \begin{figure}
% 	 	\includegraphics[width = .75\textwidth]{imgs/distribution.png}
% 	 	\caption{Distribution of values for resulting reconstructions over 36 experiments}
%	 \end{figure}
%\end{frame}



%%%% BACKUPPPP
\begin{frame}{}
	\begin{center}
		\vspace{30mm}
		\Large	Gracias!
	\end{center}
%	\centering \Large
%	\emph{Gracias}
\end{frame}
\begin{frame}{Observational Model}[noframenumbering]
%\textcolor{forestgreen}{Goal:} Reconstruct signals from noisy measurements as accurately as possible. 
  Let $I_n \in \mathbb{R}^{n \times n}$ be the $n \times n$ identity matrix. Then we can write the sequence coverage matrix $A = [A_1 \; A_2] \in \mathbb{R}^{2n \times 6n}$ with:
  \renewcommand{\arraystretch}{2}
    \begin{center}
    $$ A_1 = 
    \left[
        \begin{array}{c|c|c}
            (2 \lambda_P - \varepsilon)I_n  & 0 & 0 \\ \hline
            0  & (2 \lambda_C - \varepsilon)I_n &  (2 \lambda_C - \varepsilon)I_n 
        \end{array}
    \right]
    $$
    \end{center}
    and 
    \begin{center}
    $$ A_2 = 
    \left[
        \begin{array}{c|c|c}
            (\lambda_P - \varepsilon)I_n  & 0 & 0 \\ \hline
            0  & (\lambda_C - \varepsilon)I_n &  (\lambda_C- \varepsilon)I_n 
        \end{array}
    \right]
    $$
    \end{center}
    \end{frame}
    
% %%%%%%%%%%%%%%%%%%%%%%%%%%%%%%%%%%%%%%%%%%%%
% \begin{frame}{\phantom{hello}}
% \null\vfill 
% \begin{columns}
% 	\begin{column}{0.7\textwidth}
% 		\begin{center}
% 			\font\endfont = cmss10 at 8mm
% 			\color{black}
% 			\endfont%
% 			I. Previous Work
% 		\end{center}    
% 	\end{column}
% \end{columns}
% \vfill\null
% \end{frame}
% %%%%%%%%%%%%%%%%%%%%%%%%%%%%%%%%%%%%%%%%%%%%
% \begin{frame}[fragile]{I. Image Denoising Using Recurrent Neural Network}
% 	\only<1>{\textbf{\textcolor{red}{Goal:}} Denoise image using a recurrent neural network (RNN) \textbf{without training data}\footnote[1]{A. Ho, J. Alvarez and R. F. Marcia, ``Convolution Padding in Recurrent Neural Networks for Image Denoising with Limited Data,” Submitted to 2021 Asilomar Conference on Signals, Systems, and Computers.}.}
% 	\only<2>{\textbf{\textcolor{red}{Goal:}} Denoise image using a recurrent neural network (RNN) \textbf{without training data}\footnote[2]{D. Ulyanov, A. Vedaldi, and V. Lempitsky, ``Deep Image Prior," 2018.}.}
	
% 	%\only<1->{\textbf{\textcolor{blue}{Goal:}} Denoise image using a recurrent neural network (RNN) \textbf{without training data}\footnotemark[1].}
	 
% 	 \only<1>{
% 	 	\bigskip
% 	 	\begin{figure}
% 	 	\includegraphics[width = .75\textwidth]{figs/denoise.png}
% 	 \end{figure}}
	 
% 	\only<2>{\begin{center}
% 	\begin{columns}
% 		\begin{column}{0.3\textwidth}
% 			\vspace{5mm}
% 			\scalebox{0.8}{
% 				\begin{tabular}{cl}
% 					$z_{noise}$ & = \quad Gaussian noise \\
% 					$z_t$ & = \quad Current input \\
% 					$\epsilon$ & = \quad weighted sum parameter \\
% 					$U_\phi$ & = \quad Encoder with parameters $\phi$ \\
% 					$V_\theta$ & = \quad Decoder with parameters $\theta$ \\
% 					$f(z_t)$ & = \quad Reconstruction at time $t$
% 				\end{tabular}
% 			}
% 		\end{column}
% 		\hspace{5mm}
% 		\begin{column}{0.65\textwidth}
% 			\vspace{-5mm}
% 			\begin{figure}
% 				\includegraphics[width=.75\linewidth]{figures/rnn_final3.png}
% 			\end{figure}
% 		\end{column}
% 	\end{columns}
% 	\end{center}}

% \end{frame}
% %%%%%%%%%%%%%%%%%%%%%%%%%%%%%%%%%%%%%%%%%%%%
% \begin{frame}[fragile]{I. Results}
% 	\only<1>{
% 	\begin{center}
% 		\resizebox{0.85\textwidth}{!}{%
% 			\begin{figure*}
% 				\centering
% 				\begin{tabular}{ccccccccc}
% 					&
% 					Ground Truth &
% 					Noisy &
% 					Reconstruction &
% 					\phantom{Edge Padding} &
% 					\phantom{Reflection Padding}
% 					\\[.2cm]
% 					\parbox[t]{2mm}{\rotatebox[origin=c]{90}{Image 1 \hspace{-3.5cm} }} \!\!\!\!
% 					&
% 					\includegraphics[width=3.0cm]{figures/image_1} \!\!\!\!\!
% 					&
% 					\includegraphics[width=3.0cm]{figures/noisy_image_1} \!\!\!\!\!
% 					&
% 					\includegraphics[width=3.0cm]{figures/unpad_image_1} \!\!\!\!\!
% 					&
% 					\phantom{\includegraphics[width=3.0cm]{figures/pad_image_1} \!\!\!\!\!}
% 					&
% 					\phantom{\includegraphics[width=3.0cm]{figures/reflect_image_0}}\\
% 					&
% 					&
% 					&
% 					\phantom{\Red{SSIM$_F = 0.8800$}} &
% 					\phantom{\Blue{SSIM$_F = 0.9077$}} &
% 					\phantom{\Blue{SSIM$_F = 0.9018$}} \\
% 					& 
% 					&
% 					&
% 					\phantom{\Blue{SSIM$_C = 0.8818$}} &
% 					\phantom{\Red{SSIM$_C = 0.8276$}} &
% 					\phantom{\Red{SSIM$_C = 0.7996$}} 
% 					\\[.2cm]
% 					\parbox[t]{2mm}{\rotatebox[origin=c]{90}{Image 2 \hspace{-3.5cm} }} \!\!\!\!
% 					&
% 					\includegraphics[width=3.0cm]{figures/image_13} \!\!\!\!\!
% 					&
% 					\includegraphics[width=3.0cm]{figures/noisy_image_13} \!\!\!\!\!
% 					&
% 					\includegraphics[width=3.0cm]{figures/unpad_image_13} \!\!\!\!\!
% 					&
% 					\phantom{\includegraphics[width=3.0cm]{figures/pad_image_13} \!\!\!\!\!}
% 					&
% 					\phantom{\includegraphics[width=3.0cm]{figures/reflect_image_13}}\\
% 					& 
% 					&
% 					&
% 					\phantom{\Red{SSIM$_F = 0.8162$}}&
% 					\phantom{\Blue{SSIM$_F = 0.8210$}} &
% 					\phantom{\Blue{SSIM$_F = 0.8305$}} \\
% 					& 
% 					&
% 					&
% 					\phantom{\Blue{SSIM$_C = 0.7778$}} &
% 					\phantom{\Red{SSIM$_C = 0.7434$}} &
% 					\phantom{\Red{SSIM$_C = 0.7881$}} 			
% 				\end{tabular}
% 		\end{figure*}}}
	
% 	\only<2->{	\begin{center}
% 			\resizebox{0.85\textwidth}{!}{%
% 				\begin{figure*}
% 					\centering
% 					\begin{tabular}{ccccccccc}
% 						&
% 						Ground Truth &
% 						Noisy &
% 						Unpadded &
% 						Edge Padding &
% 						Reflection Padding
% 						\\[.2cm]
% 						\parbox[t]{2mm}{\rotatebox[origin=c]{90}{Image 1 \hspace{-3.5cm} }} \!\!\!\!
% 						&
% 						\includegraphics[width=3.0cm]{figures/image_1} \!\!\!\!\!
% 						&
% 						\includegraphics[width=3.0cm]{figures/noisy_image_1} \!\!\!\!\!
% 						&
% 						\includegraphics[width=3.0cm]{figures/unpad_image_1} \!\!\!\!\!
% 						&
% 						\includegraphics[width=3.0cm]{figures/pad_image_1} \!\!\!\!\!
% 						&
% 						\includegraphics[width=3.0cm]{figures/reflect_image_0}\\
% 						&
% 						&
% 						&
% 						\Red{SSIM$_F = 0.8800$} &
% 						\Blue{SSIM$_F = 0.9077$} &
% 						\Blue{SSIM$_F = 0.9018$} \\
% 						& 
% 						&
% 						&
% 						\Blue{SSIM$_C = 0.8818$} &
% 						\Red{SSIM$_C = 0.8276$} &
% 						\Red{SSIM$_C = 0.7996$} 
% 						\\[.2cm]
% 						\parbox[t]{2mm}{\rotatebox[origin=c]{90}{Image 2 \hspace{-3.5cm} }} \!\!\!\!
% 						&
% 						\includegraphics[width=3.0cm]{figures/image_13} \!\!\!\!\!
% 						&
% 						\includegraphics[width=3.0cm]{figures/noisy_image_13} \!\!\!\!\!
% 						&
% 						\includegraphics[width=3.0cm]{figures/unpad_image_13} \!\!\!\!\!
% 						&
% 						\includegraphics[width=3.0cm]{figures/pad_image_13} \!\!\!\!\!
% 						&
% 						\includegraphics[width=3.0cm]{figures/reflect_image_13}\\
% 						& 
% 						&
% 						&
% 						\Red{SSIM$_F = 0.8162$}&
% 						\Blue{SSIM$_F = 0.8210$} &
% 						\Blue{SSIM$_F = 0.8305$} \\
% 						& 
% 						&
% 						&
% 						\Blue{SSIM$_C = 0.7778$} &
% 						\Red{SSIM$_C = 0.7434$} &
% 						\Red{SSIM$_C = 0.7881$} 			
% 					\end{tabular}
% 		\end{figure*}}}
% 	\footnotetext[1]{SSIM: Structural Similarity Index Measure}
% 	\end{center}
% \vspace{-3mm}
%  Conclusions:
% 	\begin{itemize}
% 		\only<1->{\item Utilized RNN architecture for image denoising without any information from the true image.}
% 		\only<2->{\item The overall restoration of the image was improved by adding padding.}
% 	\end{itemize}
% \end{frame}
% %%%%%%%%%%%%%%%%%%%%%%%%%%%%%%%%%%%%%%%%%%%
% \begin{frame}{\phantom{hello}}
% 	\null\vfill 
% 	\begin{columns}
% 		\begin{column}{0.7\textwidth}
% 			\begin{center}
% 				\font\endfont = cmss10 at 8mm
% 				\color{black}
% 				\endfont%
% 				II. Ongoing Work
% 			\end{center}    
% 		\end{column}
% 	\end{columns}
% 	\vfill\null
% \end{frame}
% %%%%%%%%%%%%%%%% SAR STUFF %%%%%%%%%%%%%%%%%%%%
% \begin{frame}{II. Synthetic Aperture Radar (SAR)}
% \textbf{Setup:} In SAR typically a single transmitter/receiver is used to collect the scattered electromagnetic field over a synthetic aperture that is created by a moving platform\footnote[1]{ M. Cheney, ``A mathematical tutorial on synthetic aperture radar," 2001.}.
% \vspace{-3mm}
% \begin{columns}
% \column{0.5\textwidth}
% \begin{align*}
% \mathbf{f}(t) & = \text{broadband pulse} \\
% t &= \text{fast time parameter} \\
% \vr(s) &= \text{location of the platform} \\
% s &= \text{slow time parameter} \\
% \textcolor{blue}{\brho(\vy)} &= \textcolor{blue}{\text{unknown reflectivity}} \\
% \textcolor{forestgreen}{\bd(s,t)} &= \textcolor{forestgreen}{\text{SAR observation data}} \\
% \end{align*}

% \column{0.5\textwidth}
% \begin{figure}
% 	\centering
% 	\includegraphics[width=0.7\textwidth]{images/SAR_edit1}
% 	\label{fig:saredit1}
% \end{figure}
% \end{columns}
% \end{frame}
% %%%%%%%%%%%%%%%%%%%%%%%%%%%%%%%%%%%%%%%%%%%
% \begin{frame}{II. Direct Scattering Problem}
% The SAR process can be modeled using a scalar wave equation, which is discritized and reduces the imaging problem as
% \begin{equation*}
% \bA \textcolor{blue}{\brho} = \textcolor{forestgreen}{\bd},
% \end{equation*}
% where 
% \begin{align*}	
% \bA &\in \CC^{N\times K} \text{is the sensing matrix} \\
% \textcolor{blue}{\brho} &\in \mathbb{C}^K \text{is the reflectivity vector} \\
% \textcolor{forestgreen}{\bd} &\in \CC^{N} \text{is the SAR observation data.}
% \end{align*}

% \pause

% \bigskip

% We assume that $N \gg K$ and $\bA$ has full-column rank.
% \end{frame}
% %%%%%%%%%%%%%%%%%%%%%%%%%%%%%%%%%%%%%%%%%%%%%%%
% \begin{frame}{II. Inverse Scattering Problem}
% 	\textbf{Approximating }$\brho$: The least-squares solution is given by
	
% 	\begin{equation*}
% 	\hbrho_{\ell_2} = (\bA^H\! \bA)^{-1} \bA^H\bd,
% 	\end{equation*}
	
% 	where $\bA$ is full column rank and $\bA^H$ is the complex conjugate-transpose of $\bA$. 
	
% 	\bigskip
	
% 	\bigskip
	
% 	\pause 
% 	A common choice in discretizing the imaging window leads to the matrix $\bA^H\!\bA$ being close to diagonal\footnote[1]{L.Borcea et al, ``Synthetic aperture imaging of directional and frequency dependent reflectivity," 2016.}, which justifies the SAR inversion formula,
	
% 	\begin{equation*}
% 	\hbrho_{\text {SAR}} = \bA^H \bd.
% 	\end{equation*}
	
% \end{frame}

% %\subsection{Machine Learning for SAR}
% %%%%%%%%%%%%%%%%%%%%%%%%%%%%%%%%%%%%%%%%%%%
% \begin{frame}[fragile]{II.A: Machine Learning for SAR}
% 	Our method\footnotemark[1] is based on the use of a \textbf{neural network} in order to discover a mapping between a set \textcolor{blue}{\textbf{reflectivities} $\{\boldsymbol{\varrho}_p\}$} and a set of \textcolor{forestgreen}{\textbf{measurements} $\{\boldsymbol{d}_p\}$}.
	
% 	\only<1>{\begin{center}
% 		\includegraphics[scale=.3]{images/fc_network2}
% 	\end{center}}
	
% 	\only<2>{\begin{center}
% 		\includegraphics[scale=.3]{images/fc_network_cross2}
% 	\end{center}
% 	\vspace{-2mm}
% 		\textcolor{blue}{\underline{\textbf{Our Approach:}}}
% 		\begin{itemize}
% 			{\item \textcolor{blue}{One} fully connected layer is used.}
% 			{\item \textcolor{blue}{No activation} function is used.}
% 			{\item The focus is on the recovery of the \textcolor{blue}{transformation}.}
% 	\end{itemize}}

% 	\footnotetext[1]{O. DeGuchy, J. Alvarez, A. D. Kim, R. F. Marcia and C. Tsogka, ``Forward and inverse scattering in synthetic aperture radar using machine learning," Proceedings of SPIE 2020, Applications of Machine Learning.}
% \end{frame}
% %%%%%%%%%%%%%%%%%%%%%%%%%%%%%%%%%%%%%%%%%%%
% \begin{frame}[fragile]{II.A: Training Process - Direct Scattering}
% 	\textbf{Recovering forward mapping:} Given a set of \Blue{\textbf{reflectivities} $\{\boldsymbol{\varrho}_p\}$} and their corresponding \textcolor{forestgreen}{\textbf{measurements} $\{\boldsymbol{d}_p\}$} we recover the sensing matrix $\boldsymbol{A}$ by
	
% 	\only<1>{\begin{center}
% 		\includegraphics[scale=.44]{images/pre-process}
% 	\end{center}}
% 	\only<2->{\begin{center}
% 		\includegraphics[scale=.35]{images/learn_AH}
% \end{center}}
% 	\footnotesize
% 	\begin{enumerate}
% 		\only<1->{\item \Blue{\textbf{Reshape}} the reflectivity into its vector format and separate the real and imaginary values.} 
% 		\only<2->{\item Perform a \Blue{\textbf{left multiply}} by the sensing matrix initialized with \Blue{\textbf{random values}}.}
% 		\only<3->{\item \Blue{\textbf{Compare}} the output measurement with the target measurements using the \Blue{\textbf{Mean Squared Error (MSE)}} and minimize using a variant of \Blue{\textbf{gradient descent}}.}
% 	\end{enumerate}
% \end{frame}

% \begin{frame}[fragile]{II.A): Training Process - Inverse Scattering}
% 	\textbf{Recovering inverse mapping:} Given a set of \textcolor{forestgreen}{\textbf{measurements} $\{\boldsymbol{d}_p\}$}  and their corresponding \Blue{\textbf{reflectivities} $\{\boldsymbol{\varrho}_p\}$} we recover the pseudo-inverse $\boldsymbol{B}$ by
	
% 	\begin{center}
% 		\includegraphics[scale=.35]{images/learn_BL}
% 	\end{center}
% 	\footnotesize
% 	\begin{enumerate}
% 		\item \Blue{\textbf{Reshape}} the reflectivity into its vector format and separate the real and imaginary values.
% 		\item Perform a \Blue{\textbf{left multiply}} by the sensing matrix initialized with \Blue{\textbf{random values}}.
% 		\item \Blue{\textbf{Compare}} the output measurement with the target measurements using the \Blue{\textbf{Mean Squared Error (MSE)} and minimize using a variant of \Blue{\textbf{gradient descent}}.}
% 	\end{enumerate}
% \end{frame}
% %%%%%%%%%%%%%%%%%%%%%%%%%%%%%%%%%%%%%%%%%%%
% \begin{frame}[fragile]{II.A: Experiment 1}
% 	Low-resolution training data for low-resolution reconstructions.
% 	\vspace{-3mm}
% 	\begin{center}	
% 		\resizebox{\textwidth}{!}{%
% 			\begin{tabular}{c}
% 				Target Image\\
% 				\includegraphics[width=3.75cm]{figs/extended_clean_true_7000}\\
% 			\end{tabular}
% 			\begin{tabular}{ccccccccc}
% 				&
% 				$\hat{\boldsymbol{\varrho}}_{\text{SAR}} = \bA^H \bd$ &
% 				$\boldsymbol{\hat{\varrho}}_{L} = \bA_L^H \bd$ &
% 				$\boldsymbol{\hat{\varrho}}_{I} = \bB_L \bd$ \\[.2cm]
% 				\parbox[t]{2mm}{\rotatebox[origin=c]{90}{Noiseless \hspace{-3.25cm} }} \!\!\!\!
% 				&
% 				\includegraphics[width=3.75cm]{figs/extended_clean_KM_7000} \!\!\!\!\!\!\!\!
% 				&
% 				\includegraphics[width=3.75cm]{figs/extended_clean_AH_7000} \!\!\!\!\!\!\!\!
% 				&
% 				\includegraphics[width=3.75cm]{figs/extended_clean_BL_7000}\\
% 				& \textbf{MSE = 0.01467} &
% 				\textbf{MSE = 0.01465} &
% 				\Blue{\textbf{MSE = 0.00143}}
% 				\\[.2cm]
% 				\parbox[t]{2mm}{\rotatebox[origin=c]{90}{SNR = 5db \hspace{-3.5cm} }} \!\!\!\!
% 				&
% 				\includegraphics[width=3.75cm]{figs/extended_5db_KM_7000} \!\!\!\!\!\!\!\!
% 				&
% 				\includegraphics[width=3.75cm]{figs/extended_5db_AH_7000} \!\!\!\!\!\!\!\!
% 				&
% 				\includegraphics[width=3.75cm]{figs/extended_5db_BL_7000}\\
% 				& \textbf{MSE = 0.03069} &
% 				\textbf{MSE = 0.03009} &
% 				\Blue{\textbf{MSE = 0.00985}}
% 				\\[.2cm]
% 			\end{tabular}
% 		}
% 	\end{center}	
% \end{frame}
% %%%%%%%%%%%%%%%%%%%%%%%%%%%%%%%%%%%%%%%%%%%
% \begin{frame}[fragile]{II.A: Experiment 2}
% 	Low-resolution training data for high-resolution reconstructions.
% 	\vspace{-3mm}
% 	\begin{center}
% 		\resizebox{\textwidth}{!}{%
% 			\begin{tabular}{c}
% 				Target Image\\
% 				\includegraphics[width=3.75cm]{figs/point_clean_true_1000}\\
				
% 			\end{tabular}
% 			\begin{tabular}{ccccccccc}
% 				&
% 				$\hat{\boldsymbol{\varrho}}_{\text{SAR}} = \bA^H \bd$ &
% 				$\boldsymbol{\hat{\varrho}}_{L} = \bA^H_L \bd$ &
% 				$\boldsymbol{\hat{\varrho}}_{I} = \bB \bd$ \\[.2cm]
% 				\parbox[t]{2mm}{\rotatebox[origin=c]{90}{Noiseless \hspace{-3.25cm} }} \!\!\!\!
% 				&
% 				\includegraphics[width=3.75cm]{figs/point_clean_KM_1000} \!\!\!\!\!\!\!\!
% 				&
% 				\includegraphics[width=3.75cm]{figs/point_clean_AH_1000} \!\!\!\!\!\!\!\!
% 				&
% 				\includegraphics[width=3.75cm]{figs/point_clean_BL_1000}\\
% 				&
% 				\textbf{MSE = 0.00614} &
% 				\textbf{MSE = 0.00604} &
% 				\Blue{\textbf{MSE = 0.00309}}
% 				\\[.2cm]
% 				\parbox[t]{2mm}{\rotatebox[origin=c]{90}{SNR - 5db \hspace{-3.5cm} }} \!\!\!\!
% 				&
% 				\includegraphics[width=3.75cm]{figs/point_5db_KM_1000}  \!\!\!\!\!\!\!\!
% 				&
% 				\includegraphics[width=3.75cm]{figs/point_5db_AH_1000} \!\!\!\!\!\!\!\!
% 				&
% 				\includegraphics[width=3.75cm]{figs/point_5db_BL_1000}\\
% 				&
% 				\textbf{MSE = 0.00930} &
% 				\textbf{MSE = 0.00910} &
% 				\Blue{\textbf{MSE = 0.00479}}
% 				\\[.2cm]
% 		\end{tabular}}
% 	\end{center}
% \end{frame}

% %\subsection{Image Classification in SAR}
% %%%%%%%%%%%%%%%%%%%%%%%%%%%%%%%%%%%%%%%%%%%
% \begin{frame}{II.B: Image Classification in SAR}
% \textcolor{red}{\textbf{Goal:}} Given SAR observation data $\bd$, \textcolor{blue}{\textbf{classify}} the original unknown reflectivities $\brho$ using three different reconstruction methods\footnotemark[1].

% \vspace{-1.5cm}
% \begin{columns}
% 	\column{0.2\textwidth}
% 		\begin{tabular}{l}
% 			\hspace{-1cm} \includegraphics[width=4cm]{images/cars/SAR_clean.pdf}
% 		\end{tabular}
% 	\column{0.8\textwidth}
% 		\textcolor{blue}{\textbf{Method I:}} Classify using $\bd$ (raw SAR data).
% \end{columns}
% \vspace{-3.5cm}
% \begin{columns}
% 	\column{0.2\textwidth}
% 	\begin{tabular}{l}
% 		 \hspace{-1cm} \includegraphics[width=4cm]{images/cars/AH_clean.pdf}
% 	\end{tabular}
% 	\column{0.8\textwidth}
% 		\textcolor{orange}{\textbf{Method II:}} Classify using reconstructions, $ \hbrho_{\text {SAR}} = \bA^H \bd$.
% \end{columns}
% \vspace{-3.5cm}
% \begin{columns}
% 	\column{0.2\textwidth}
% 	\begin{tabular}{l}
% 		\hspace{-1cm} \includegraphics[width=4cm]{images/cars/BL_clean.pdf}
% 	\end{tabular}
% 	\column{0.8\textwidth}
% 		\textcolor{forestgreen}{\textbf{Method III:}} Classify using reconstructions from the learned approximate inverse, $ \hbrho_{I} =  \bB_L\bd$.
% \end{columns}
% \vspace{-2cm}
% \footnotetext[1]{J. Alvarez, O. DeGuchy and R. F. Marcia, ``Image Classification In Synthetic Aperture Radar Using Reconstruction From Learned Inverse Scattering," 2020 IEEE Geoscience and Remote Sensing Symposium (IGARSS).}
% \end{frame}

% %%%%%%%%%%%%%%%%%%%%%%%%%%%%%%%%%%%%%%%%%%%%%%%%%%%
% \begin{frame}{II.B: Experiment 1: Multi-Layer Perceptron (MLP)}

% \begin{figure}
% 	\centering
% 	\includegraphics[width=.8\linewidth]{images/MLP}
% \end{figure}
% Summary:
% \begin{itemize}
% 	\item \textcolor{forestgreen}{Method III} (learned inverse) has the highest classification accuracy.
% 	\item As noise levels increase, \textcolor{forestgreen}{Method III} continues to outperform the other two methods.
% \end{itemize}
% \end{frame}
% %%%%%%%%%%%%%%%%%%%%%%%%%%%%%%%%%%%%%%%%%%%
% \begin{frame}{II.B: Experiment 2: Convolutional Neural Network (CNN) }

% \begin{figure}
% 	\centering
% 	\includegraphics[width=.8\linewidth]{images/CNN}
% \end{figure}
% Summary:
% \begin{itemize}
% 	\item \textcolor{forestgreen}{Method III} has the highest classification accuracy.
% 	\item CNN achieves a higher classification accuracy over MLP.
% \end{itemize}
% \end{frame}

% %\subsection{Exploring Learned ``Inverse" in SAR}
% %%%%%%%%%%%%%%%%%%%%%%%%%%%%%%%%%%%%%%%%%%%
% \begin{frame}{II.C: SAR Extension (ongoing work)}
% \textcolor{red}{\textbf{Goal:}} Improve reconstruction of SAR images using information from the learned sensing matrix and convolution neural networks. \\
% \bigskip
% \textcolor{blue}{\textbf{Approach:}}
% \begin{enumerate}
% 	\item Learn $\bA_L$ and compute $\boldsymbol{\hat{\varrho}}_{L} = \bA^H_L \bd$.
% 	\item Denoise reconstruction using CNN.
% \end{enumerate}

% \begin{figure}[H]
% 	\begin{subfigure}{0.33\textwidth}
% 		\centering
% 		\includegraphics[width=\textwidth]{figs/sar_target.png}
% 	\end{subfigure}%
% 	\begin{subfigure}{0.33\textwidth}
% 		\centering
% 		\includegraphics[width=\textwidth]{figs/sar_input.png}
% 	\end{subfigure}
% 	\begin{subfigure}{0.33\textwidth}
% 		\centering
% 		\includegraphics[width=\textwidth]{figs/sar_recon.png}
% 	\end{subfigure}
% \end{figure}
% \begin{center}
% 	\vspace{-1.5mm}
% 	\small{Figure: Preliminary Results}
% \end{center}

% \footnotetext[1]{Submitted abstract to 2022 SPIE Optics and Photonics.}
% \end{frame}

% %%%%%%%%%%%%%%%%% Projection Stuff %%%%%%%%%%%%%%%%%%%

% %\subsection{Motivation: Structural Variants}
% \begin{frame}{II.D: Structural Variant Genome Detection}
% 	\begin{figure}
% 		\includegraphics[width=8cm]{imgs/SV.pdf}
% 	\end{figure}
% 	\begin{itemize}
% 		\item Structural variants (SVs) are deletions, inversions, duplications,
% 		and  transpositions in genomic sequences
% 		\item Structural variants are rare occurrences
% 		\item \textcolor{red}{\textbf{Goal:}} Improve structural variant predictions
% 	\end{itemize}
% 	\footnotetext[1]{M. Spence et al, ``Detecting novel structural variants in genomes by leveraging parent-child relatedness," 2018.}
% \end{frame}
% %%%%%%%%%%%%%%%%%%%%%%%%%%%%%%%%%%%%%%%%%%%%%%%%%%%%%%%%%
% \begin{frame}{II.D: Sequencing data}
% 	\begin{figure}
% 		\centering
% 		\includegraphics[width=0.65\textwidth]{imgs/mycov} \quad   
% 	\end{figure}
	
% 	\bigskip
	
% 	\begin{center}
% 		\begin{tabular}{|c|c|}
% 			\hline 
% 			\rowcolor{blue} \White{High coverage} & \qquad \White{Low coverage} \qquad \phantom{.} \\
% 			\hline \hline
% 			Less susceptible to error & Noisier data \\
% 			Expensive & Cheaper\\
% 			\hline
% 		\end{tabular}
		
% 	\end{center}
	
% 	\pause
	
% 	\textbf{Our regime:} Low-coverage sequencing data from related individuals.
% 	\footnotetext[1]{M. Spence et al, ``Detecting novel structural variants in genomes by leveraging parent-child relatedness," 2018.}
% \end{frame}
% %%%%%%%%%%%%%%%%%%%%%%%%%%%%%%%%%%%%%%%%%%%%%%%%%%%%%%%%%
% \begin{frame}{II.D: Detection Framework}
% 	True SV signal for each individual is a binary vector, $\vec{f}$:
% 	\begin{eqnarray*}
% 		f_i &=& 1 \text{\ \ if SV is present at location $i$}\\
% 		f_i &=& 0 \text{\ \ otherwise}
% 	\end{eqnarray*}
	
% 	\pause
	
% 	Assume we have sequencing data for two parents and one child:
% 	\begin{figure}
% 		\includegraphics[width=4.5cm]{imgs/SV_binary2}
% 	\end{figure}
	
% 	\vspace{-4cm}
	
% 	Parent 1
	
% 	\vspace{1cm}
	
% 	Parent 2
	
% 	\vspace{1cm}
	
% 	Child 
% 	\footnotetext[1]{M. Spence et al, ``Detecting novel structural variants in genomes by leveraging parent-child relatedness," 2018.}
% \end{frame}
% %%%%%%%%%%%%%%%%%%%%%%%%%%%%%%%%%%%%%%%%%%%%%%%%%%%%%%%%%
% \begin{frame}{II.D: Relatedness}
% 	Assumptions:
% 	\begin{itemize}
% 		\item If both parents have an SV, the child must have the SV.
% 		\item If neither parents have an SV, the child cannot have the SV.
% 		\item If one parent has an SV, the child may have the SV.
% 	\end{itemize}\pause
	
% 	\bigskip
	
% 	Mathematically, we write these as constraints:
% 	\begin{equation*}
% 	\begin{array}{rcccl}
% 	\vec{f}_{p_1} + \vec{f}_{p_2}  - \vec{1} & \le & \vec{f}_c & \le & 
% 	\vec{f}_{p_1} + \vec{f}_{p_2} \\[.1cm]
% 	\vec{0} & \le & \vec{f}_{p_1} & \le & \vec{1} \\[.1cm]
% 	\vec{0} & \le & \vec{f}_{p_2} & \le & \vec{1} \\[.1cm]
% 	\vec{0} & \le & \vec{f}_{c} & \le & \vec{1} \\ 
% 	\end{array}
% 	\end{equation*}
% 	\footnotetext[1]{M. Spence et al, ``Detecting novel structural variants in genomes by leveraging parent-child relatedness," 2018.}
% \end{frame}
% %%%%%%%%%%%%%%%%%%%%%%%%%%%%%%%%%%%%%%%%%%%%%%%%%%%%%%%%%
% %\subsection{Optimization Problem Formulation}
% \begin{frame}{II.D: Optimization framework}
% 	The optimization framework for the SV detection can be written as the following:
% 	\begin{eqnarray*}
% 		&& \underset{\vec{f} \in \Re^{3n}}{\text{minimize}} \ \quad F(\vec{f}) + \tau \| \vec{f} \|_1 \\
% 		&& \text{subject to } \ \ 
% 		\vec{f}_{p_1} + \vec{f}_{p_2}  - \vec{1} \le  \vec{f}_c  \ \le \
% 		\vec{f}_{p_1} + \vec{f}_{p_2}\\
% 		&& \phantom{\text{subject to }} \ \ 
% 		\vec{0} \ \le \ \vec{f}_{p_1}, \vec{f}_{p_2}, \vec{f}_c \ \le \ \vec{1}
% 	\end{eqnarray*}
% 	where 
% 	\begin{itemize}
% 		\item $F(\vec{f})$: statistical model for genome mapping
% 		\item $\tau\| \vec{f} \|_1 $: $\ell_1$ penalty term where $\tau > 0$ is a regularization parameter
% 	\end{itemize}
% 	\pause
% 	\medskip
% 	\textbf{Approach:}
% 	\begin{enumerate}
% 		\item Use second-order Taylor series to approximate $F(\vec{f})$
% 		\item Write as a sequence of quadratic minimization problems with  linear constraints
% 	\end{enumerate}
% 	\footnotetext[1]{M. Spence et al, ``Detecting novel structural variants in genomes by leveraging parent-child relatedness," 2018.}
% \end{frame}
% %%%%%%%%%%%%%%%%%%%%%%%%%%%%%%%%%%%%%%%%%%%%%%%%%%%%%%%%%
% \begin{frame}{II.D: Projection onto Affine Constraints}
% The sequence of quadratic minimization problems take on the following form:
% \begin{equation}
% \begin{aligned}
% &\underset{x \in \mathbb{R}^n}{\text{minimize}} \quad  \ \frac{1}{2}|| x - \hat x ||_2^2 \\
% &\text{subject to} \quad Ax \geq b\\
% \end{aligned}
% \end{equation}
% where $\hat x$ might not be feasible. \\
% \bigskip
% \only<1>{
% \begin{figure}
% 	\includegraphics[width = .4\textwidth]{figs/feasible_region}
% \end{figure}}
% \pause
% \textbf{Goal:} Solve large-scale problems efficiently.
% \begin{itemize}
% 	\item Common approaches include interior point and active set methods.
% 	\item Alternative: Apply simpler method.
% \end{itemize}
% \bigskip
% \pause
% \textbf{Approach:} 
% 	\begin{enumerate}
% 	\item Transform primal problem (Eq. 1) into a different problem with simpler constraints.
% 	\item Formulate Lagrangian function.
% \end{enumerate}
% \end{frame}
% %%%%%%%%%%%%%%%%%%%%%%%%%%%%%%%%%%%%%%%%%%%%%%%%%%%%%%%%%	
% \begin{frame}{II.D: Lagrangian Function}
% The \textbf{Lagrangian function} $L : \mathbb{R}^n \times \mathbb{R}^m \rightarrow \mathbb{R}$ associated with the primal problem is given by
% \begin{equation*} \label{eq:Lagrangian}
% L(x,\lambda) = \underbrace{\frac{1}{2}|| x - \hat x ||_2^2}_{f(x)} - \ \lambda^T \underbrace{(Ax - b)}_{c(x)},
% \end{equation*}
% with Lagrange multipliers $\lambda \in \mathbb{R}^m$ where $\lambda \geq 0$.\\
% \vspace{\baselineskip}
% \pause
% \begin{columns}
% 	\column{0.6\textwidth}
% 	\textbf{\textcolor{red}{Intuition:}}
% 	\begin{itemize}
% 		\item For $\lambda \geq 0$ and $c(x) \geq 0$: $\lambda^T c(x) \geq 0$.
% 		\item Thus, $L(x,\lambda) \leq f(x)$.
% 		\item $\underset{x, \lambda}{\text{Maximize }} L(x,\lambda) \Longleftrightarrow \underset{x}{\text{Minimize }} f(x)$.
% 	\end{itemize}
	
% 	\column{0.4\textwidth}
% 	\vspace{-11mm}
% 	\begin{figure}
% 		\centering
% 		\includegraphics[width=\textwidth]{figs/primal-dual2}
% 	\end{figure}
% \end{columns}

% %\textcolor{red}{Intuition:}
% %\begin{itemize}
% %	\item For $\lambda \geq 0$ and feasible $x$, i.e., $c(x) \geq 0$, we have $\lambda^T c(x) \geq 0$.
% %	\item Thus, $L(x,\lambda) \leq f(x)$.
% %	\item Maximizing $L(x,\lambda)$ might be easier than minimizing $f(x)$.
% %\end{itemize}
% \end{frame}
% %%%%%%%%%%%%%%%%%%%%%%%%%%%%%%%%%%%%%%%%%%%%%%%%%%%%%%%%%
% \begin{frame}{II.D: Lagrangian Dual Function}
% The \textbf{Lagrangian dual function} is defined as $g : \mathbb{R}^m\rightarrow \mathbb{R}$ where
% \begin{align*}
% g(\lambda) & = \inf_{x \in \mathbb{R}^n} L(x,\lambda) = \inf_{x \in \mathbb{R}^n} \left(f(x) - \lambda^T c(x)\right)
% \end{align*}
% Defining $g(\lambda)$: Setting $\nabla L_x(x,\lambda) = 0$, we obtain
% $$ x^* = \hat x + A^T \lambda.$$
% Then substituting $x^*$ into $L(x,\lambda)$, we find 
% $$g(\lambda) = \inf_{x \in \mathbb{R}^n} L(x,\lambda) = -\frac{1}{2} \lambda^T AA^T\lambda - \lambda^T(A\hat{x} - b).$$
% \end{frame}
% %%%%%%%%%%%%%%%%%%%%%%%%%%%%%%%%%%%%%%%%%%%%%%%%%%%%%%%%%
% \begin{frame}{II.(d): Dual Problem}
% 	Maximizing $g(\lambda)$ is equilavent to minimizing $-g(\lambda)$. Thus, the \textbf{dual problem} associated with our primal problem is
% 	\begin{equation*}
% 	\begin{aligned}
% 	&\underset{\lambda \in \mathbb{R}^m}{\text{minimize}} \  \quad \tilde g(\lambda) =  \frac{1}{2} \lambda^T AA^T\lambda + \lambda^T (A \hat x - b)\\
% 	&\text{subject to} \quad \lambda \geq 0.
% 	\end{aligned}
% 	\end{equation*}\\
% 	Then using the solution $\lambda^*$ we can define $x^*$ from $ x^* = \hat x + A^T \lambda$. \\
% 	\vspace{8mm}
% 	\only<1>{The \textbf{duality gap} is the difference between the primal and dual solutions,
% 	$$f(x^*) - g(\lambda^*)$$
% 	where $x^*$ and $\lambda^*$ are the minimizers of the primal and dual problems, respectively.}
% 	\only<2>{\textbf{Approach: }Use projected gradient method where 
% 	\begin{enumerate}
% 		\item Perform gradient descent on the current solution
% 		\item Project it back onto the constraint set
% 	\end{enumerate}}
% \end{frame}
% %%%%%%%%%%%%%%%%%%%%%%%%%%%%%%%%%%%%%%%%%%%%%%%%%%%%%%%%%
% \begin{frame}{II.D: Proposed Method}
% 	\only<1->{\textbf{Projected Gradient Method\footnotemark[1]:}
% 	\begin{figure}
% 	\includegraphics[width = .5\textwidth]{figs/projected_grad_dual}
% 	\end{figure}}
% 	\only<2>{The update on $\lambda$ is given by:
% 	\vspace{-2mm}
% 	$$\lambda_{k+1} = [\lambda_k - \alpha \nabla \tilde{g}(\lambda_k)]_+ 	\vspace{-3mm}$$
% 	where 
% 	\begin{itemize}
% 		\item $\alpha$ is the step size and calculated using a backtracking line search
% 		\item $[ \ \cdot \ ]_+  = \max \{0,\cdot\}$
% 	\end{itemize}}
% \footnotetext{J. Nocedal and S. Wright. Numerical optimization.}
% \end{frame}
% %%%%%%%%%%%%%%%%%%%%%%%%%%%%%%%%%%%%%%%%%%%%%%%%%%%%%%%%%
% \begin{frame}{II.D: Numerical Experiment}
% 	We construct the set of constraints $Ax \geq b$, where $A \in \mathbb{R}^{m \times n}$ and $b \in \mathbb{R}^m$ where 
% 	\begin{itemize}
% 		\item $m = 200$
% 		\item $n = 60$
% 	\end{itemize}
% 	\medskip
% 	\textbf{Stopping criterion:}
% 		\begin{itemize}
% 		\item $|| x_k - x_{k-1} ||_2 \ < \epsilon$ where $\epsilon = 10^{-5}$ 
% 		\item Max number of iterations is $k = 100$
% 	\end{itemize}
% 	\medskip
% 	\textbf{Metrics:}
% 		\begin{itemize}
% 		\item Calculate duality gap: $f(x) - g(\lambda)$.
% 		\item Check infeasibility: Calculate the residual, given by $Ax - b$. Specifically,
% 		$$\sum_{i \in \{ i: (Ax - b)_i < 0\} }^m (Ax - b)_i.$$
% 	\end{itemize}
% \end{frame}
% %%%%%%%%%%%%%%%%%%%%%%%%%%%%%%%%%%%%%%%%%%%%%%%%%%%%%%%%%
% \begin{frame}{II.D: Results}
% \begin{figure}[H]
% 	\begin{subfigure}{0.5\textwidth}
% 		\centering
% 		Duality Gap
% 		\includegraphics[width=\textwidth]{imgs/method2_dual_2.png}
% 	\end{subfigure}%
% 	\begin{subfigure}{0.5\textwidth}
% 		\centering
% 		Infeasibility
% 		\includegraphics[width=\textwidth]{imgs/method2_infeas_2.png}
% 	\end{subfigure}
% \end{figure}

% \begin{center}
% 	\vspace{-1.5mm}
% 	\small{Figure: $m = 200$ and $n = 60$}
% \end{center}

% \vspace{-3mm}
% Summary:
% \begin{itemize}
% 	\item Duality gap and infeasibility converged to zero.
% 	\item Proposed method was able to find a solution for high dimensional problems.
% \end{itemize}
% \end{frame}
% %%%%%%%%%%%%%%%%%%%%%%%%%%%%%%%%%%%%%%%%%%%%%%%%%%%%%%%%%
% %\subsection{Future Work}
% \begin{frame}{II.D: Extensions}
% 	\textbf{Goal 1:} Solve sparse reconstruction problem\footnotemark[1] given by
% 	\begin{equation*} \label{eq:gpsr}
% 		\hspace{-1.8cm}
% 		\begin{aligned}
% 			&\underset{x}{\text{minimize}} \quad \ \frac{1}{2} \norm{y - Ax}_2^2 + \tau \norm{x}_1 \\
% 			&\text{subject to} \quad E x \ge d
% 		\end{aligned}
% 	\end{equation*}
% 	with linear constraints.
% 	\flushleft{\textbf{Goal 2:} Solve the constrained sparse Poisson inverse problem\footnotemark[2] given by}
% 	\begin{equation*} \label{eq:gpsr}
% 		\hspace{-1.8cm}
% 		\begin{aligned}
% 			&\underset{f}{\text{minimize}} \quad \ 1^T Af - \sum_{i=1}^m y_i \log (e_i^T Af + \beta) + \tau \norm{x}_1 \\
% 			&\text{subject to} \quad f \ge 0.
% 		\end{aligned}
% 	\end{equation*}
% 	\flushleft{\textbf{Goal 3:} Apply developed method to structural variant detection framework.}
% 	\footnotetext[1]{M.A.T. Figueiredo et al, ``Gradient projection for sparse reconstruction: Application to compressed sensing and other inverse problems," 2007.}
% 	\footnotetext[2]{Z.T. Harmany et al, ``This is spiral-tap: Sparse poisson intensity reconstruction algorithms—theory and practice," 2011.}
% \end{frame}
% %%%%%%%%%%%%%%%%%%%%%%%%%%%%%%%%%%%%%%%%%%%%%%%%%%%
% \begin{frame}{\phantom{hello}}
% 	\null\vfill 
% 	\begin{columns}
% 		\begin{column}{0.7\textwidth}
% 			\begin{center}
% 				\font\endfont = cmss10 at 8mm
% 				\color{black}
% 				\endfont%
% 				III. Future Work
% 			\end{center}    
% 		\end{column}
% 	\end{columns}
% 	\vfill\null
% \end{frame}
% %%%%%%%%%%%%%%%%%%%%%%%%%%%%%%%%%%%%%%%%%%%%%%%%%%%
% \begin{frame}{III. Trust-region for Machine Learning}
% Machine learning problems, such as text or image classification, are often nonlinear and nonconvex unconstrained optimization problems given by
% \begin{align*}
% \underset{w}{\text{minimize}} \quad & f(w) = \frac{1}{n} \sum_{i=1}^n \mathcal{L}(h(x_i, w), y_i) 
% \end{align*}
% where 
% \begin{align*}	
% 	n:& \text{ number of training samples} \\
% 	w:& \text{ parameter of the mapping function} \\
% 	x_i:& \text{ feature vector of the $i^{th}$ training sample} \\
% 	y_i:& \text{ corresponding real-valued observation} \\
% 	\mathcal{L}:& \text{ loss function}
% \end{align*}

% \textcolor{red}{\textbf{Goal:}} Improve the optimization method and apply to machine learning problem in limited data regimes.
% \end{frame}
% %%%%%%%%%%%%%%%%%%%%%%%%%%%%%%%%%%%%%%%%%%%%%%%%%%%%%%%%%
% \begin{frame}{III. Common Methods}
% The most common approaches in machine learning include:
% \smallskip
% \begin{enumerate}
% 	\item Stochastic gradient descent (SGD)
% 	\begin{itemize}
% 		\item Approximates the gradient using a \textbf{random subset} of the data.
% 		\item \Blue{\textbf{Advantage:}}  Computationally inexpensive.
% 		\item \Red{\textbf{Disadvantage:}} Manual hyperparameter tuning.
% 	\end{itemize}
% 	\medskip
% 	\pause
% 	\item Quasi-Newton approach using the limited-memory Broyden-Fletcher-Goldfarb-Shannon (L-BFGS) update
% 	\begin{itemize}
% 		\item Quasi-Newton methods use first-order information to \textbf{approximate the second derivative}.
% 		\item The L-BFGS updates generate a sequence of \textbf{positive definite matrices}.
% 		\item \Blue{\textbf{Advantage:}} Does not require manual parameter tuning. 
% 		\item \Red{\textbf{Disadvantage:}} Approximate an indefinite matrix as positive definite.
% 	\end{itemize}
% \end{enumerate}
% \smallskip
% \pause
% \textbf{Proposed Approach:}
% \begin{itemize}
% \item Symmetric Rank-1 Update: Not guaranteed to be positive definite. \\
% \item Trust-region: Suitable for indefinite approximations.
% \end{itemize}
% \end{frame}
% %%%%%%%%%%%%%%%%%%%%%%%%%%%%%%%%%%%%%%%%%%%%%%%%%%%%%%%%%
% \begin{frame}{III. Symmetric Rank-1 (SR1) Update}
% A typical update of a quasi-Newton method is as follows:
% \begin{align*}
% w_{k+1} = w_k + \eta_k p_k \quad \text{where} \quad p_k = -B_k^{-1} \nabla f(w_k)
% \end{align*}
% \pause
% The symmetric rank-1 (SR1) update given by 
% \begin{align*}
% B_{k+1} = B_k + \frac{(y_k - B_k s_k)(y_k - B_k s_k)^T}{(y_k - B_k s_k)^Ts_k}
% \end{align*}
% where $s_k = w_{k+1} - w_k$ and $y_k = \nabla f(w_{k+1}) - \nabla f(w_k)$.
% \bigskip
% \begin{itemize}
% 	\item Unlike BFGS, the update is not guaranteed to be positive definite.
% 	\item This allows for more accurate approximations of the true Hessian matrix.
% 	\item Due to the indefinite Hessian approximation, we are not guaranteed a descent direction. 
% 	\item Thus, we utilize a trust-region framework over a line search framework. 
% \end{itemize}
% \end{frame}
% %%%%%%%%%%%%%%%%%%%%%%%%%%%%%%%%%%%%%%%%%%%%%%%%%%%%%%%%%
% \begin{frame}{III. Line search vs. Trust-Region}
% 	Two major techniques for unconstrained optimization.\\
% 	\bigskip
% 	\textbf{Line search methods: }
% 	\begin{enumerate}
% 		\item Generate a search direction
% 		\item Find an appropriate step size to move in search direction
% 	\end{enumerate}
% 	\bigskip
% 	\textbf{Trust-region methods: }
% 	\begin{enumerate}
% 		\item Define a region in which the quadratic model approximates the objective function well
% 		\item Choose direction which minimizes the model in this region
% 	\end{enumerate}
% \end{frame}
% %%%%%%%%%%%%%%%%%%%%%%%%%%%%%%%%%%%%%%%%%%%%%%%%%%%%%%%%%
% \begin{frame}{III. Trust-Region Subproblem}
% 	At each iteration of the trust-region method, we solve what is know as the trust-region subproblem given by
% 	\begin{align*} \label{eq:subproblem} 
% 	&\underset{p}{\text{minimize}} \quad \ Q(p) = g^Tp + \frac{1}{2}p^T B_k p \\
% 	&\text{subject to} \quad \norm{p} \leq \delta. \nonumber
% 	\end{align*}
% 	where $g$ is the gradient of $f$ and $B$ is an approximation of the Hessian.  \\
% 	\bigskip
% 	\pause
% 	\textbf{Formulation using 2-norm:}
% 	\begin{itemize}
% 		\item Global minimizer can be characterized \footnote[1]{D.M. Gay, ``Computing optimal locally constrained steps. SIAM Journal on Scientific and Statistical Computing," 1981.}\footnote[2]{J.J. Moré et al, ``Newton’s method," 1984.}.
% 		\item There is no closed form solution to this subproblem.
% 	\end{itemize}
% \end{frame}
% %%%%%%%%%%%%%%%%%%%%%%%%%%%%%%%%%%%%%%%%%%%%%%%%%%%%%%%%%
% \begin{frame}{III. SR1 Trust-Region Approach}
% 	The SR1 trust-region approach has shown to improve performance on MNIST classification\footnote[1]{J.B. Erway, et al, ``Trust-region algorithms for training responses: machine learning methods using indefinite hessian approximations," 2020.}.  \\ 
% 	\pause
% 	\medskip
% 	\textbf{Motivation:}
% 		\begin{enumerate}
% 		\item Improving choice of initial $B_0$.
% 		\item Solving trust-region subproblem efficiently.
% 	\end{enumerate}
%  	\pause
% 	\medskip
% 	\textbf{Proposed Approaches:}
% 	\begin{enumerate}
% 		\item Alternative choice for $B_0$.
% 		\item Use of a particular norm to obtain closed form solutions to the trust-region subproblem.
% 	\end{enumerate}
	
% \end{frame}
% %%%%%%%%%%%%%%%%%%%%%%%%%%%%%%%%%%%%%%%%%%%%%%%%%%%%%%%%%
% \begin{frame}{Improving L-SR1 Trust-Region Framework}
% 	\textcolor{forestgreen}{\textbf{Dense Initialization}} \\
% 	\smallskip
% 	\only<1-2>{The SR1 update can be written as the following:}
% 	\only<3-9>{The SR1 update can be written as the following\footnotemark[1]:}
% 	\begin{align*}
% 	\only<1>{\hspace{15mm}B_k = & \ B_{k-1} + \underbrace{\frac{(y_{k-1} - B_{k-1} s_{k-1})(y_{k-1} - B_{k-1} s_{k-1})^T}{(y_{k-1} - B_{k-1} s_{k-1})^Ts_{k-1}}}_{\text{rank 1}} \phantom{\underbrace{\sum_{j = 0}^{k-1}}_{\text{rank $k$}} \underbrace{\Bigg[ \Psi_k \Bigg]}_{\text{rank $k$}}}}
% 	\only<2>{\hspace{-5.9mm} \ B_k = & \ \underbrace{B_0}_{\gamma_k I} + \underbrace{\sum_{j = 0}^{k-1} \frac{(y_j - B_j s_j)(y_j - B_j s_j)^T}{(y_j - B_j s_j)^Ts_j}}_{\text{rank $k$}}\phantom{\underbrace{\sum_{j = 0}^{k-1}}_{\text{rank $k$}} \underbrace{\Bigg[ \Psi_k \Bigg]}_{\text{rank $k$}}}}
% 	\only<3-4>{\hspace{-12.8mm} \  \ B_k = & \ \underbrace{B_0}_{\gamma_k I} + \underbrace{\Bigg[ \Psi_k \Bigg] [ \ M_k \ ] [ \quad \Psi_k^T \quad ]}_{\text{rank $k$}} \phantom{\underbrace{\sum_{j = 0}^{k-1}}_{\text{rank $k$}} \underbrace{\Bigg[ \Psi_k \Bigg]}_{\text{rank $k$}}}}
% 	\only<5>{\hspace{19.8mm} B_k = & \ \underbrace{B_0}_{\gamma_k I} + \underbrace{\Bigg[\Psi_k \Bigg] [ \ R^{-1} \ ] [ \ W \ ] }_{P_{\parallel}} [ \ \hat \Lambda \ ] \underbrace{[ \ W^T \ ] [ \ R^{-T} \ ] [ \quad \Psi_k^T \quad ]}_{P_{\parallel}^T} \phantom{\underbrace{\sum_{j = 0}^{k-1}}_{\text{rank $k$}} \underbrace{\Bigg[ \Psi_k \Bigg]}_{\text{rank $k$}}}}
% 	\only<6>{\hspace{-12.3mm} B_k = & \ \underbrace{B_0}_{\gamma_k I}  +  \hspace{0.7mm} \Bigg[ P_{\parallel} \Bigg] [ \ \hat \Lambda \ ] [ \quad P_{\parallel}^T \quad ] \phantom{\underbrace{\sum_{j = 0}^{k-1}}_{\text{rank $k$}} \underbrace{\Bigg[ \Psi_k \Bigg]}_{\text{rank $k$}}}} 
% 	\only<7>{\hspace{-8.5mm} B_k = & \ \underbrace{B_0}_{\gamma_k I} +  \begin{bmatrix}  P_{\parallel} & P_{\perp} \end{bmatrix} \begin{bmatrix}  \hat \Lambda & 0 \\  0 & 0 \end{bmatrix} \begin{bmatrix}  P_{\parallel}^T \\ P_{\perp}^T \end{bmatrix} \phantom{\underbrace{\sum_{j = 0}^{k-1}}_{\text{rank $k$}} \underbrace{\Bigg[ \Psi_k \Bigg]}_{\text{rank $k$}}}}
% 	\only<8->{\hspace{-8mm} B_k = &  \begin{bmatrix}  P_{\parallel} & P_{\perp} \end{bmatrix} \begin{bmatrix}  \hat \Lambda+\gamma_k I & 0 \\  0 & \gamma_k I \end{bmatrix} \begin{bmatrix}  P_{\parallel}^T \\ P_{\perp}^T \end{bmatrix} \phantom{\underbrace{\sum_{j = 0}^{k-1}}_{\text{rank $k$}} \underbrace{\Bigg[ \Psi_k \Bigg]}_{\text{rank $k$}}}}
% 	\end{align*}
% 	\only<3-4>{where 
% 			\vspace{-3mm}
% 			\begin{align*}
% 			\Psi_k &= Y_k - B_0 S_k \in \mathbb{R}^{n \times k} \\
% 			Y_k &= [ \ y_0 \ \ y_1 \ \cdots \ y_{k-1} \ ] \in \mathbb{R}^{n \times k} \\
% 			S_k &= [ \ s_0 \ \ s_1 \ \cdots \ s_{k-1} \ ] \in \mathbb{R}^{n \times k}\\ 
% 			M_k &= k \times k \ \text{matrix}
% 			\end{align*}}
% 	\only<4>{We compute $\Psi_k = QR$ and $M_k = W \hat \Lambda W^T$.}
% 	\only<7->{\vspace{-8.5mm} \\ where $P_{\perp}$ is the orthogonal complement of $P_{\parallel}$.}
% 	\only<9>{
% 		\smallskip
% 	\begin{itemize}
% 		\item Since $P P^T = I$, we write $B_0 = \gamma_k I = \gamma_k P P^T = \gamma_k P_{\parallel}P_{\parallel}^T + \gamma_k P_{\perp}P_{\perp}^T.$
% 		\item Define two different $\gamma$ parameters\footnotemark[2] such that $$\hat{B_0} = \gamma_k P_{\parallel}P_{\parallel}^T + \gamma_k^{\perp} P_{\perp}P_{\perp}^T.$$
% 		\item \textcolor{blue}{\textbf{Approach:}} Apply to L-SR1 and investigate alternative approaches to choosing $\gamma_k^{\perp}$.
% 	\end{itemize}}

% 	\only<3-9>{
% 	\footnotetext[1]{R.H. Byrd et al, ``Representations of quasi-Newton matrices and their use in limited-memory methods", 1994.}}

% 	\only<9>{\footnotetext[2]{J. Brust et al, ``A dense initialization for limited-memory quasi-newton methods," 2019.}}
% \end{frame}	
% %%%%%%%%%%%%%%%%%%%%%%%%%%%%%%%%%%%%%%%%%%%%%%%%%%%%%%%%%
% \begin{frame}{III. Improving L-SR1 Trust-Region Framework}
% 	\textcolor{forestgreen}{\textbf{Shape-changing norm}} \\
% 	\smallskip
% 	Solve the trust-region subproblem defined by the shape-changing norm\footnotemark[1]:
% 	\begin{align*}
% 	&\underset{p}{\text{minimize}} \quad \ Q(p) = g^Tp + \frac{1}{2}p^T B_k p \\
% 	&\text{subject to} \quad  \norm{p}_{P,\star} \leq \delta_k \nonumber
% 	\end{align*}
% where 
% 	\begin{equation*}
% 	\norm{p}_{P,\star} \triangleq \text{max} \left(  \normr{P_{\parallel}^T p}_{\star}, \normr{P_{\perp}^T p}_2  \right)
% 	\end{equation*}
% \vspace{-2mm}
% 	\begin{itemize}
% 		\item Decouples into two separate problems, each with a closed form solution.
% 		\item \textcolor{blue}{\textbf{Approach:}} Apply to L-SR1 trust-region framework.
% 	\end{itemize}
% \footnotetext[1]{O. Burdakov, et al, ``On efficiently combining limited-memory and trust-region techniques," 2017.}
% \end{frame}	
% %%%%%%%%%%%%%%%%%%%%%%%%%%%%%%%%%%%%%%%%%%%%%%%%%%%%%%%%%
% \begin{frame}{III. Trust-Region: Example Application}
% \textcolor{red}{\textbf{Goal:}} Classify MRI images of brains to aid in prevention of Alzheimer’s disease (AD) using deep learning techniques and apply proposed optimization methodology\footnote[1]{A. De Luna, ``Classification of Mild Cognitive Impairment Under Data Limitations Using Deep
% 	Learning Techniques," 2020.}.
% \smallskip
% \begin{itemize}
% 	\item The images focus on the identification of mild cognitive impairment (MCI).
% 	\item MCI is divided into early and late MCI (EMCI and LMCI), which are both intermediate stage before the diagnosis of AD.
% 	\item We want to classify between normal brains and those with EMCI and LMCI.
% 	\item Moreover, this classification was studied under data limitations.
% \end{itemize}
% \end{frame}
% %%%%%%%%%%%%%%%%%%%%%%%%%%%%%%%%%%%%%%%%%%%%%%%
% \begin{frame}{Summary}
% 	\begin{enumerate}
% 		\item \textbf{Imaging Denoising Using Recurrent Neural Networks }(complete)
% 		\smallskip
% 		\item \textbf{Synthetic Aperture Radar}
% 		\begin{itemize}
% 			\item Machine Learning for SAR (complete)
% 			\item SAR Image Classication (complete)
% 			\item Extension (ongoing)
% 		\end{itemize}
% 		\item \textbf{Projected Gradients}
% 		\begin{itemize}
% 			\item Sparse Reconstructions (Goal 1)
% 			\item Poisson Reconstructions (Goal 2)
% 			\item SV Detection (Goal 3)
% 		\end{itemize}
% 		\item \textbf{Proposed Research}
% 		\begin{itemize}
% 			\item Dense Initialization
% 			\item Shape-Changing Norm
% 			\item Applications
% 		\end{itemize}
% 	\end{enumerate}
% \end{frame}
% %%%%%%%%%%%%%%%%%%%%%%%%%%%%%%%%%%%%%%%%%%%%%%%%%%%
% \begin{frame}{\phantom{hello}}
% 	\null\vfill 
% 	\begin{columns}
% 		\begin{column}{0.7\textwidth}
% 			\begin{center}
% 				\font\endfont = cmss10 at 8mm
% 				\color{black}
% 				\endfont%
% 				IV. Accomplishments
% 			\end{center}    
% 		\end{column}
% 	\end{columns}
% 	\vfill\null
% \end{frame}	
% %%%%%%%%%%%%%%%%%%%%%%%%%%%%%%%%%%%%%%%%%%%%%%%%%%%
% \begin{frame}{IV. Course  and Special Requirements}
% 	\begin{figure}
% 		\includegraphics[width=\textwidth]{figs/all_requirements.png}
% 	\end{figure}
% \end{frame}
% %%%%%%%%%%%%%%%%%%%%%%%%%%%%%%%%%%%%%%%%%%%%%%%%%%%%%%%%%
% \begin{frame}{IV. Publications}
% 	\begin{enumerate}
% 		\item O. DeGuchy, J. Alvarez, A. D. Kim, R. F. Marcia and C. Tsogka, ``Forward and inverse scattering in synthetic aperture radar using machine learning," Proceedings of SPIE 2020, Applications of Machine Learning.
% 		\vspace{\baselineskip}
% 		\item J. Alvarez, O. DeGuchy and R. F. Marcia, ``Image Classification In Synthetic Aperture Radar Using Reconstruction From Learned Inverse Scattering," 2020 IEEE Geoscience and Remote Sensing Symposium (IGARSS).
% 		\vspace{\baselineskip}
% 		\item A. Ho, J. Alvarez and R. F. Marcia, ``Convolution Padding in Recurrent Neural Networks for Image Denoising with Limited Data,” Submitted to 2021 Asilomar Conference on Signals, Systems, and Computers.
% 		\vspace{\baselineskip}
% 		\item J. Alvarez, A. D. Kim, R. F. Marcia and C. Tsogka, ``Convolution Padding in Recurrent Neural Networks for Image Denoising with Limited Data,” Abstract submitted to 2022 SPIE Optics and Phonotics.
% 	\end{enumerate}
% \end{frame}
% %%%%%%%%%%%%%%%%%%%%%%%%%%%%%%%%%%%%%%%%%%%%%%%%%%%%%%%%%
% \begin{frame}{IV. Presentations}
% 		\begin{enumerate}
% 		\item ``Forward and inverse scattering in synthetic aperture radar using machine learning" 
% 			\begin{itemize}
% 				\item SPIE Optics + Photonics Conference (August 2020)
% 			\end{itemize}
% 		\vspace{\baselineskip}
% 		\item ``Image Classification In Synthetic Aperture Radar Using Reconstruction From Learned Inverse Scattering"
% 			\begin{itemize}
% 				\item SIAM Annual - Workshop Celebrating Diversity (July 2020)
% 				\item IEEE Geoscience and Remote Sensing Symposium	(September 2020)
% 				\item ACM Richard Tapia Conference (September 2020)
% 				\item SACNAS National Diversity in STEM Conference  (October 2020)
% 				\item SIAM CSE - Broader Engagements Program (March 2021)
% 			\end{itemize}
% 		\vspace{\baselineskip}
% 		\item ``Convolution Padding in Recurrent Neural Networks for Image Denoising with Limited Data”
% 			\begin{itemize}
% 				\item ACM Richard Tapia Conference (September 2021)
% 				\item SACNAS National Diversity in STEM Conference (October 2021)
% 				\item Asilomar Conference on Signals, Systems, and Computers (November 2021)
% 			\end{itemize}
% 	\end{enumerate}
% \end{frame}
% %%%%%%%%%%%%%%%%%%%%%%%%%%%%%%%%%%%%%%%%%%%%%%%%%%%%%%%%%
% \begin{frame}{IV. Accomplishments}
% 	\textbf{Summer Programs/Internships:}
% 	\begin{itemize}
% 		\item 2019 ATR Center Summer Internship 
% 		\item 2020 ATR Center Summer Internship
% 		\item 2020 Data Science Challenge
% 		\item 2021 LLNL Data Science Summer Institute Internship
% 	\end{itemize}
% 	\vspace{\baselineskip}
% 	\textbf{Conference Scholarships:}
% 	\begin{itemize}
% 		\item 2020 SACNAS Travel Scholarship
% 		\item 2020 Tapia Travel Scholarship
% 		\item 2020 Grace Hopper Celebration Scholar
% 		\item 2021 SIAM CSE Broader Engagement Program
% 		\item 2021 SACNAS Travel Scholarship
% 		\item 2021 Tapia Travel Scholarship
% 	\end{itemize}
% \end{frame}
% %%%%%%%%%%%%%%%%%%%%%%%%%%%%%%%%%%%%%%%%%%%%%%%%%%%%%%%%%
% \begin{frame}{IV. Accomplishments and Service}
% 	\begin{tabular}{ll}
% 		\textbf{2019-2020:}    &   NSF-RTG Fellowship\\
% 		&   UC Merced SIAM Student Chapter Secretary  \\ 
% 		& Applied Mathematics Excellence in Service Award \\
% 		& Applied Mathematics Summer Travel Fellowship \\
% 		& Cal State Fresno SIAM Grad School Panelist \\ \\
% 		\textbf{2020-2021:} & UC Merced SIAM Student Chapter Officer Vice President \\ 
% 		& Applied Mathematics Excellence in Service Award \\
% 		& Led Machine Learning Mini Course \\
% 		& Applied Mathematics Internship Recognition Award \\ \\
% 		\textbf{2021-2022:} & NRT-IAS Fellowship \\
% 		& UC Merced SIAM Student Chapter Officer President \\
% 		& Cal State Fresno SIAM Grad School Panelist
% 	\end{tabular}	
% \end{frame}
% %%%%%%%%%%%%%%%%%%%%%%%%%%%%%%%%%%%%%%%%%%%%%%%%%%
% \begin{frame}{\phantom{hello}}
% 	\null\vfill 
% 	\begin{columns}
% 		\begin{column}{0.7\textwidth}
% 			\begin{center}
% 				\font\endfont = cmss10 at 8mm
% 				\color{black}
% 				\endfont%
% 				V. Timeline
% 			\end{center}    
% 		\end{column}
% 	\end{columns}
% 	\vfill\null
% \end{frame}	
% %%%%%%%%%%%%%%%%%%%%%%%%%%%%%%%%%%%%%%%%%%%%%%%%%%%
% \begin{frame}{V. Timeline}
% \begin{table}[h!]
% 	\begin{center}
% 		\scalebox{0.9}{
% 		\begin{tabular}{ll}
% 			\hline
% 			Time                           &   Tasks  \\
% 			\hline \\
% 			Spring 2022         &  Complete Qualifying Exam \hspace{6cm} \\
% 			&     Write manuscript for SAR Extension \\
% 			& 	  \ \ - Submit manuscript to SPIE (July 2022) \\ 
% 			&     Write manuscript for Goal 1 of projected gradients \\
% 			&     \ \ - Submit manuscript to ICASSP (October 2022)\\ \\
% 			Summer 2022      &   Summer Internship - Lawrence Livermore National Laboratory \\ \\
% 			\hline
% 		\end{tabular}}
% 	\end{center}
% \end{table}
% \end{frame}
% %%%%%%%%%%%%%%%%%%%%%%%%%%%%%%%%%%%%%%%%%%%%%%%%%%%%%%%%%
% \begin{frame}{V. Timeline}
% \begin{table}[h!]
% 	\begin{center}
% 		\scalebox{0.9}{
% 		\begin{tabular}{ll}
% 			\hline
% 			Time                           &   Tasks     \\
% 			\hline \\
% 			Fall   2022            &   Complete Goal 2 and Goal 3 for projected gradients \hspace{2.3cm} \\ 
% 			&   \ \ - Write and submit manuscripts for Goals 2 and 3  \\ \\
% 			Spring 2023         &   Implement SR1 trust-region approach in Python \\
% 			&    \ \ - Complete initial draft of manuscript \\ \\
% 			Summer 2023      &   Summer Internship   \\ \\
% 			\hline
% 		\end{tabular}}
% 	\end{center}
% \end{table}
% \end{frame}
% %%%%%%%%%%%%%%%%%%%%%%%%%%%%%%%%%%%%%%%%%%%%%%%%%%%%%%%%%
% \begin{frame}{V. Timeline}
% \begin{table}[h!]
% 	\begin{center}
% 		\scalebox{0.9}{
% 		\begin{tabular}{ll}
% 			\hline
% 			Time                           &   Tasks     \\
% 			\hline \\
% 			Fall   2023            &	Finalize draft of trust-region and submit manuscript \hspace{2.3cm} \\
% 			&   Apply proposed trust-region method to applications \\
% 			&   Begin initial draft of dissertation manuscript   \\ \\
% 			Spring 2024 \hspace{2.5mm}        &   Complete dissertation manuscript  \\
% 			&   Defend dissertation \\ \\
% 			\hline
% 		\end{tabular}}
% 	\end{center}
% \end{table}
% \end{frame}
% %%%%%%%%%%%%%%%%%%%%%%%%%%%%%%%%%%%%%%%%%%%%%%%%%%%
% \begin{frame}
% \null\vfill 
%         \begin{columns}
%             \begin{column}{0.7\textwidth}
%               \begin{center}
%                 \font\endfont = cmss10 at 10mm
%                 \color{black}
%                 \endfont%
%                 Thank you!
%               \end{center}    
%             \end{column}
%           \end{columns}
%           \vfill\null
% \end{frame}
% %%%%%%%%%%%%%%%%%%%%%%%%%%%%%%%%%%%%%%%%%%%%%%%%%%%%%%%%%
% \begin{frame}[fragile]{Recurrent Neural Networks}
% 	\begin{figure}
% 		\centering
% 		\includegraphics[width = .9\textwidth]{figures/rnn_unrolled.png}
% 	\end{figure}
	
% 	\begin{itemize}
% 		\item Use feedback loops
% 		\item Contain internal states (serves as memory)
% 		\item Used for tasks with sequential data such as speech recognition, target tracking, etc.
% 		\item Typically not used for image processing tasks
% 	\end{itemize}
% \end{frame}
% %%%%%%%%%%%%%%%%%%%%%%%%%%%%
% \begin{frame}[fragile]{Recurrent Neural Networks}
% 	A typical RNN structure can be described by the following:
% 	\begin{align*}
% 	\begin{split}
% 	\textbf{h}_t &= \text{tanh}(\textbf{\textcolor{dkgreen}{W}}\textbf{h}_{t-1} + \textbf{\textcolor{dkgreen}{U}}\textcolor{blue}{\textbf{x}_t} + \textbf{\textcolor{orange}{b}}) \\
% 	\textcolor{mauve}{\hat{\textbf{y}}_t} &= \text{softmax}(\textbf{\textcolor{dkgreen}{V}}\textbf{h}_t + \textbf{\textcolor{orange}{c}})
% 	\end{split}
% 	\end{align*}
% 	where $\textbf{\textcolor{dkgreen}{W}}, \textbf{\textcolor{dkgreen}{U}}$ and $\textbf{\textcolor{dkgreen}{V}}$ are weight matrices, $\textbf{\textcolor{orange}{b}}$ and $\textbf{\textcolor{orange}{c}}$ are bias vectors, $\textcolor{blue}{\textbf{x}_t}$ is the input vector and $\textbf{h}_t$ is the current state used to find the predicted output vector $\textcolor{mauve}{\hat{\textbf{y}}_t}$ where time goes from $t = 0$ to $t = T$.
% \end{frame}
% %%%%%%%%%%%%%%%%%%%%%%%%%%%%%%%%%%%%%%%%%%%%%%%%%%%%%%%%%%%%%%%%%%%%%%%%%%%%%%%%%%%%%%%%%%%%%
% \begin{frame}[fragile]{Autoencoders}
% 	A common neural network architecture used for denoising is an autoencoder, which is composed of three parts: encoder, decoder and latent space representation.
	
% 	\vspace{\baselineskip}
% 	\begin{figure}
% 		\centering
% 		\includegraphics[width = \textwidth]{figures/denoising_autoencoder.png}
% 	\end{figure}
	
% \end{frame}
% %%%%%%%%%%%%%%%%%%%%%%%%%%%%%%%%
% \begin{frame}[fragile]{Evaluation Criterion}
% 	After recovering the restored image we evaluate the denoising approach using the \textbf{structural similarity index (SSIM)} between the recovered image and the true image, which is given by
% 	\vspace{3mm}
% 	\begin{equation*}
% 	\text{SSIM}(x, y) = \frac{(2\mu_{x}\mu_{y} + c_1)(2\sigma_{xy} +c_2)}{(\mu_{x}^2 + \mu_{y}^2 + c_1)(\sigma_{x}^2 + \sigma_{y}^2 + c_2)}
% 	\end{equation*} \\
% 	\vspace{3mm}
% 	where $\mu_{x}$ and $\mu_{y}$ are the averages of each input respectively, $\sigma_{x}$ and $\sigma_{y}$ are the associated variances, and $c_1$ and $c_2$ are variables used to stabilize the denominator.  
% \end{frame}
% %%%%%%%%%%%%%%%%%%%%%%%%%%%%%%%%%%%%%%%%%%%%%%%%%%%%%%%%%%%%%%%%%%%%%%%%%%%%%%%%%%%%%%%%%%%%%
% \begin{frame}[fragile]{Numerical Experiments: Padding}
% 	\begin{enumerate}
% 		\item [$\bullet$] Preliminary results suggested that we did not have enough information along the edges and corners of an image
% 		\item [$\bullet$] We investigated various types of padding to introduce more information along the edges
% 	\end{enumerate}
% 	\vspace{\baselineskip}
% 	\begin{center}
% 		\resizebox{.9\textwidth}{!}{%
% 			\begin{figure*}
% 				\begin{center}
% 					\hspace{-9mm}
% 					\begin{tabular}{cccccccc}
% 						Original \!\!\!\!\!\! &
% 						0-Padding \!\!\!\!\!\! &
% 						Edge \!\!\!\!\!\! &
% 						Reflection \!\!\!\!\!\! \\
% 						\includegraphics[width=2.7cm]{figures/natural.png} \!\!\!\!\!\!
% 						&
% 						\includegraphics[width=2.7cm]{figures/zero_padding.png} \!\!\!\!\!\!
% 						&
% 						\includegraphics[width=2.7cm]{figures/edge_padding.png} \!\!\!\!\!\!
% 						&
% 						\includegraphics[width=2.7cm]{figures/reflective_padding.png} \!\!\!\!\!\!
% 					\end{tabular}
% 					% 	\caption{
% 					% 		CIFAR-10 images used in our numerical experiments.  Images 1-5 and the corresponding results are presented in detail in Fig.\ 3.
% 					% 	}
% 					% 	\label{fig:images}
% 				\end{center}
% 		\end{figure*}}
% 	\end{center}
% \end{frame}
% %%%%%%%%%%%%%%%%%%%%%%%%%%%%%%%%%%%%%%%%
% \begin{frame}{Optimization framework}
% 	Solve
% 	\begin{eqnarray*}
% 		&& \underset{\vec{f} \in \Re^{3n}}{\text{minimize}} \ \quad F_{\text{NegBin}}(\vec{f}) + \tau \| \vec{f} \|_1 \\
% 		&& \text{subject to } \ \ 
% 		\vec{f}_{p_1} + \vec{f}_{p_2}  - \vec{1} \le  \vec{f}_c  \ \le \
% 		\vec{f}_{p_1} + \vec{f}_{p_2}\\
% 		&& \phantom{\text{subject to }} \ \ 
% 		\vec{0} \ \le \ \vec{f}_{p_1}, \vec{f}_{p_2}, \vec{f}_c \ \le \ \vec{1}
% 	\end{eqnarray*}
% 	Use a second-order Taylor series expansion to approximate $F_{\text{NegBin}}(\vec{f})$ at iterate $\vec{f}^k$: \\ 
% 	$F^k(\vec{f}) \approx F(\vec{f}^k) + (\vec{f} - \vec{f}^k)^T \nabla F(\vec{f}^k) + \frac{\alpha_k}{2} \| \vec{f} - \vec{f}^k  \|_2^2$ \\
% 	\vspace{\baselineskip}
% 	Solve the quadratic subproblem,
% 	\begin{eqnarray*}
% 		\vec{f}^{k+1}
% 		&=& \underset{\vec{f} \in \Re^{3n}}{\text{arg min}} \quad \ \  \frac{1}{2} \| \vec{f} - \vec{s}^k \|_2^2 
% 		+ \frac{\tau}{\alpha_k} \| \vec{f} \|_1 \\
% 		&& \text{subject to } \ \ 
% 		\vec{f}_{p_1} + \vec{f}_{p_2}  - \vec{1} \le  \vec{f}_c  \ \le \
% 		\vec{f}_{p_1} + \vec{f}_{p_2}\\
% 		&& \phantom{\text{subject to }} \ \ 
% 		\vec{0} \ \le \ \vec{f}_{p_1}, \vec{f}_{p_2}, \vec{f}_c \ \le \ \vec{1}		
% 	\end{eqnarray*}
% 	where $\vec{s}^k = \vec{f}^k - \frac{1}{\alpha_k} \nabla F_{\text{NegBin}}(\vec{f}^k)$ and 
% 	$\alpha_k$ is the learning rate.
% 	\flushright{\scriptsize{(Banuelos et al, 2018)}}
% \end{frame}
% %%%%%%%%%%%%%%%%%%%%%%%%%%%%%%%%%%%%%%%%%%%%%%%%%%%%%%%%%
% \begin{frame}{Optimization framework}
% 	Solve
% 	\vspace{-.8cm}
% 	\begin{eqnarray*}
% 		&&\underset{\vec{f} \in \Re^{3n}}{\text{arg min}} \quad \ \  \frac{1}{2} \| \vec{f} - \vec{s}^k \|_2^2 
% 		+ \frac{\tau}{\alpha_k} \| \vec{f} \|_1 \\
% 		&& \text{subject to } \ \ 
% 		\vec{f}_{p_1} + \vec{f}_{p_2}  - \vec{1} \le  \vec{f}_c  \ \le \
% 		\vec{f}_{p_1} + \vec{f}_{p_2}\\
% 		&& \phantom{\text{subject to }} \ \ 
% 		\vec{0} \ \le \ \vec{f}_{p_1}, \vec{f}_{p_2}, \vec{f}_c \ \le \ \vec{1}		
% 	\end{eqnarray*}
% 	Separating the objective function, 
% 	\begin{eqnarray*}
% 		&&\underset{{f_{p_1}}, {f_{p_2}}, {f_{c}}\in \Re}{\text{minimize}} \quad \ \ \frac{1}{2} (f_{p_1} - s_{p_1})^2 + \frac{1}{2} (f_{p_2} - s_{p_2})^2 + \frac{1}{2} (f_{c} - s_{c})^2 \\ 
% 		&& \phantom{\underset{{f_{p_1}}, {f_{p_2}}, {f_{c}}\in \Re^{n}}{\text{minimize}}} \quad \ \ + \frac{\tau}{\alpha_k}|f_{p_1}| + \frac{\tau}{\alpha_k}|f_{p_2}| + \frac{\tau}{\alpha_k}|f_c|
% 		%	&& \text{subject to } \quad \ \ \vec{f}_{p_1} + \vec{f}_{p_2}  - \vec{1} \le  \vec{f}_c  \ \le \
% 		%	\vec{f}_{p_1} + \vec{f}_{p_2}\\
% 		%	&& \phantom{\text{subject to }} \quad \ \ \vec{0} \ \le \ \vec{f}_{p_1}, \vec{f}_{p_2}, \vec{f}_c \ \le \ \vec{1}		
% 	\end{eqnarray*}
% 	Completing the squares, 
% 	\begin{eqnarray*}
% 		&&\underset{{f_{p_1}}, {f_{p_2}}, {f_{c}}\in \Re}{\text{minimize}} \quad \ \ \frac{1}{2} (f_{p_1} - a)^2 + \frac{1}{2} (f_{p_2} - b)^2 + \frac{1}{2} (f_{c} - d)^2
% 		%	&& \text{subject to } \quad \ \ \vec{f}_{p_1} + \vec{f}_{p_2}  - \vec{1} \le  \vec{f}_c  \ \le \
% 		%	\vec{f}_{p_1} + \vec{f}_{p_2}\\
% 		%	&& \phantom{\text{subject to }} \quad \ \ \vec{0} \ \le \ \vec{f}_{p_1}, \vec{f}_{p_2}, \vec{f}_c \ \le \ \vec{1}		
% 	\end{eqnarray*}
% 	where $a = s_{p_1} - \frac{\tau}{\alpha_k}$,  $b = s_{p_2} - \frac{\tau}{\alpha_k}$ and $d = s_{c} - \frac{\tau}{\alpha_k}$
% 	\flushright{\scriptsize{(Banuelos et al, 2018)}}
% \end{frame}
% %%%%%%%%%%%%%%%%%%%%%%%%%%%%%%%%%%%%%%%%%%%%%%%%%%%%%%%%%
% \begin{frame}{Strong Duality}
% \textbf{Theorem:} \\
% If the objective function is quadratic convex, and the constraints are all affine, then the duality gap is always zero, provided that one of the primal or dual problems is feasible.

% \bigskip
% Our problem:
% \begin{equation*}
% \begin{aligned}
% &\underset{x \in \mathbb{R}^n}{\text{minimize}} \quad  \ \frac{1}{2}|| x - \hat x ||_2^2 \\
% &\text{subject to} \quad Ax \geq b\\
% \end{aligned}
% \end{equation*}

% \bigskip
% We meet the two conditions that guarantee the duality gap is zero. Therefore, finding a solution to the dual problems ensures we have found the solution to the primal problem.
% \end{frame}
% %%%%%%%%%%%%%%%%%%%%%%%%%%%%%%%%%%%%%%
% \begin{frame}{Backtracking Linesearch}
% 	Let $\alpha \in (0, 1/2), \beta \in (0,1)$. Starting at $t = 1$, repeat with $t = \beta t$ until 
% 	$$ f(x + t \Delta x) < f(x) + \alpha t \nabla f(x)^T \Delta x $$

% \bigskip
% Graphical Representation:
% 	\begin{figure}
% 		\includegraphics[width = .6\textwidth]{figs/linesearch2}
% 	\end{figure}
% \end{frame}
% %%%%%%%%%%%%%%%%%%%
% \begin{frame}{Proposed Method}
% 	\begin{algorithm}[H]
% 		\SetAlgoNoLine % No outside lines around blocks 
% 		\SetKwInput{init}{Initialize} % Initialization
% 		\SetNlSty{}{}{} % Linenumbering style {<font>}{<txt before>}{<txt after>}
% 		\init{ $  \lambda_0; \mu = 0.1; \beta = 0.5; k = 0$} %\;
% 		\While{not converged} 
% 		{
% 			$\nabla \tilde{g} = AA^T \lambda + r$\;
% 			$\alpha = \dfrac{||\nabla \tilde{g}(\lambda_k)||_2^2}{||A \nabla \tilde{g}(\lambda_k)||_2^2}$
% 			\nllabel{alg1:eqs} \tcc*[r]{Initialze step size} 
			
% 			\While(\tcc*[f]{Backtracking Linesearch}){true}
% 			{
% 				$\tilde \lambda = [\lambda_k - \alpha \nabla \tilde{g}(\lambda_k)]_+$\;	
% 				\If{$\tilde{g}(\tilde \lambda) \leq \tilde{g}(\lambda_{k}) - \mu \nabla \tilde{g}(\lambda_{k} - \tilde \lambda)$}{
% 					break
% 				}
% 				$\alpha = \beta * \alpha$
% 			}
% 			$\lambda_{k+1} =  \tilde \lambda$; \tcc*[f]{Accept step}
			
% 			$x_{k+1} = \hat x + A^T \lambda_{k+1}$\;
% 		}
% 		\caption{Projected Gradinet Method with Backtracking Linesearch}
% 		\label{alg1}
% 	\end{algorithm}
% \end{frame}
% %%%%%%%%%%%%%%%%%%%%%%%%%%%%%%%%%%%%%%%%%%%%%%%%%%%%%%%%%
% \begin{frame}{Compact Representation}
% Recall the SR1 update, given by
% \begin{align*}
% B_{k+1} = B_k + \frac{(y_k - B_k s_k)(y_k - B_k s_k)^T}{(y_k - B_k s_k)^Ts_k}.
% \end{align*}
% The SR1 update can be written as using the following the compact representation:
% \begin{equation*}
% B_k = B_0 + \Psi_k M_k \Psi_k^T.
% \end{equation*}
% \bigskip
% Then, given $B_0 = \gamma_k I$ and the eigendecomposition of $\Psi_k M_k \Psi_k^T$, we can write
% \begin{equation*}
% B_k = \gamma_k I + \Psi_k M_k \Psi_k^T = P\Lambda P^T
% \end{equation*}
% where
% \begin{equation*}
% P \triangleq 
% \begin{bmatrix}  
% P_{\parallel} & P_{\perp}
% \end{bmatrix}
% , \quad \hat{\Lambda} \triangleq
% \begin{bmatrix}
% \hat{\Lambda} + \gamma_k I_r & \\
% & \gamma_k I_{n-r}
% \end{bmatrix}
% \end{equation*}
% \end{frame}
% %%%%%%%%%%%%%%%%%%%%%%%%%%%%%%%%%%%%%%%%%%%%%%%%%%%%%%%%%
% \begin{frame}{Shape Changing Norm}
% 	Given the compact representation, we now consider the trust-region subproblem defined by the shape-changing infinity norm:
% 	\begin{equation*}
% 	\norm{p}_{P,\infty} \triangleq \text{max} \left(  \normr{P_{\parallel}^T p}_{\infty}, \normr{P_{\perp}^T p}_2  \right).
% 	\end{equation*}
% 	\flushright{\scriptsize{(Burdakov et al, 2017)}}
% 	\bigskip
% 	\begin{itemize}
% 		\item This norm depends on the eigenvectors of $B_k$, thus the shape of the trust region changes each time the quasi-Newton matrix is updated.
% 		\item Defining the subproblem with this norm decouples into two separate problems, each with a closed form solution.
% 		\item Recall, when the trust region subproblem is formulated using the 2-norm, we do not have a closed form solution.
% 	\end{itemize}
% \end{frame}
% %%%%%%%%%%%%%%%%%%%%%%%%%%%%%%%%%%%%%%%%%%%%%%%%%%%%%%%%%
% \begin{frame}{Dense Initialization}
% We introduced the compact representation of SR1, where the recursive formulation depends on an initialization $B_0$. \\
% \medskip
% A common choice for $B_0$ is the solution to a generalized eigenvalue problem.\\
% \medskip
% In (Brust et al, 2019), an, one that spans the columns of $P_{\parallel}$ and the other which spans the columns of $P_{\perp}$, i.e., 
% \begin{equation*}
% \hat{B_0} = \gamma_k PP^T = \gamma_k P_{\parallel}P_{\parallel}^T + \gamma_k^{\perp} P_{\perp}P_{\perp}^T.
% \end{equation*}
% The choices for $\gamma_k$ are the same as before, whereas the choice of $\gamma_k^{\perp}$ is given by 
% \begin{equation*} \label{eq:gamma_perp}
% \gamma_k^{\perp}(c,\lambda) = \lambda c \gamma_k^{\text{max}} + (1 - \lambda)\gamma_k
% \end{equation*}
% where $c \geq 1, \lambda \in [0,1],$ and
% \begin{equation*}
% \gamma_k^{\text{max}} \triangleq \underset{1\leq i \leq k}{\text{max}} \; \gamma_i.
% \end{equation*}
% \end{frame}

\end{document}



